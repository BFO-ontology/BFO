

% Introduction to this paper, what is the point, what is the scope, what are the limitations

This paper addresses a challenge that has arisen in the context of the Open Biomedical Ontologies (OBO) Foundry \cite{Smith2007} community, in particular in relation to the implementation of the Basic Formal Ontology (BFO, \cite{BFO2:Graz}) version 2, which uses the Web Ontology Language (OWL, \cite{grau2008}). 
However, the problem is of general interest to the %biomedical 
ontology 
%and data standards 
community as a whole, because it exposes and addresses a general weakness 
%of a large number of biomedical ontologies 
of ontology artefacts that
use OWL and represent time-related
%dynamic 
aspects of their domain. 
Insofar the problem is not restricted to the life science context, and it seems relevant to virtually all domains, 
in which objects change over time, or are related with each other dependent on temporal contexts.   

According to BFO and other foundational ontologies, a useful upper-level distinction is between those entities that exist in full at all times at which they exist (continuants), and those that unfold in time (occurrents). This is a distinction according to temporal mode of existence. Temporal information is also relevant when specifying the relationships between continuant entities, particularly -- as for most of the life sciences -- where continuants such as organisms and their parts, molecules, shapes, and disorders not only persist, but also continuously change over time. They continuously gain and lose parts, qualities, and dispositions. Consequently, any relational expression which makes reference to a particular continuant can have different truth values at different times and would therefore be ambiguous if time were not made explicit in the statement. 

The Relation Ontology (RO, \cite{OBO:RO}) proposed patterns for the definition of biological relations to be used throughout bio-ontologies such as the Gene Ontology \cite{go2000}, ChEBI \cite{chebinar2013} and various anatomy ontologies \cite{uberon2012}. The relations proposed include 
\mirel{part\_of}, \mirel{has\_part}, \mirel{located\_in}, \mirel{participates\_in}, and  \mirel{instance\_of}. The Relation Ontology specifies exactly how relations at the class level were to be interpreted with regard to time. That is, in order to capture the statement that every red blood cell has some oxygen molecule located in it, the class-level relation statement \mclass{RedBloodCell} \mirel{location\_of} \mclass{OxygenMolecule} (i.e. the red blood cell is the location for the oxygen molecule) should be interpreted as the following first-order logic (FOL) statement: 
\begin{equation}
\forall x \forall t : \mirel{instance\_of}(x, \mclass{RedBloodCell}, t) \rightarrow 
\exists y : \mirel{instance\_of}(y, \mclass{OxygenMolecule}, t) \wedge \mirel{located\_in}(y, x, t)
\end{equation}

This equation says that at all times that a red blood cell exists, there exists some oxygen molecule (importantly, not necessarily the same one at different times) that is located in the red blood cell \emph{at that time}. Note that instantiation of continuants such as red blood cells and oxygen molecules is also time-indexed in this pattern. 

Most of the RO relations 
%contained in the RO 
were subsequently merged into the BFO during the redesign project, such that BFO, in its second version, now provides definitions for foundational 
relations like parthood, participation, instantiation, among others. The BFO reference specification follows the RO pattern, 
% and offers the definitions of these relations, 
allocating a time index to every relation in which a continuant is one of the relata. 

The challenge is that, as can be seen in the above formula, relational expressions with a time index are \emph{ternary}, that is, they take three parameters, \emph{viz.} the two relata (which are particulars), and the time index. However, all current variants of the OWL language, for reasons of implementation and reasoning efficiency, allows only \emph{binary} relations. 
%Ternary relations cannot be directly expressed in standard OWL expressions. 
Thus, when relations such as \textbf{part\_of} are used in class level axioms
%in bio-ontologies 
in practical OWL implementations 
%(also in OBO implementations, since the semantics for the OBO language is assigned by translation to OWL), 
the time index is lost, which can have surprising reasoning consequences. 
\todo[inline, size=\tiny]{The relation between OBO and OWL should be elucidated in more detail}


The objective of this paper is the presentation and discussion of OWL design patterns that partly mitigate this limitation. It is organised as follows:\todo[inline, size=\tiny]{Revise}

In the remainder of this Background, we give an overview of the OWL language and BFO ontology into which our proposal will be embedded, and outline our examples and competency questions.
Thereafter, we survey the existing approaches to the representation of time-dependent information in
we propose and discuss different design patterns and evaluate them against the given examples in the context of their completeness, user friendliness, and performance profile for common reasoning tasks.
The concluding section presents our recommendations for the community.

\subsection*{BFO}

Basic Formal Ontology (BFO) is an upper level ontology designed to serve as a foundation from which domain ontologies can be built \cite{BFO2:Graz}. The use of shared upper level classes and relations mitigates problems that arise when several domain ontologies are used together.
%in a research pipeline.
For example, the root structure of two ontologies might partially overlap, without explicit mappings clarifying the intended interrelationships (e.g. having classes such as `process' and `event'; `object' and `thing'). Or the ontologies may make use of similarly named relations without clarifying whether the relations are intended to mean the same thing 
(e.g., $\mirel{has\_location}$, $\mirel{located\_in}$, $\mirel{included\_in}$,... ). They may further re-implement or redefine classes that are defined differently elsewhere.

BFO offers a small set of foundational classes, together with descriptive axioms, most of which make use of a set of foundational relations
%\todo[inline, size=\tiny]{``relationships'' or ``relations''? I would prefer the latter }
intended to be used in multiple ontologies. Both classes and relations are domain-independent, although it is clearly stated that BFO's focus is on the representation of natural and applied sciences, with biology and medicine as their most important representatives.  
Thus, domain-specific relations such as $\mirel{is\_tautomer\_of}$ (a chemical relation used in ChEBI) or $\mirel{is\_about}$ (in the information artifact ontology) are out of scope for BFO itself, but they should be covered by domain ontologies below BFO, defined according to the RO recommendations.  

The uppermost partition in the BFO class hierarchy reflects temporal mode of existence: \textit{continuants} are entities that (1) exist in full at any time that they exists at all, and (2) continue to exist self-identically for as long as they exist; \textit{occurrents} are entities that unfold over a period of time and thus have temporal parts. For example, you are a continuant, while your life is an occurrent.  A cell is a continuant, while the process of cell division is an occurrent.  

% TYPES AND INSTANCES
BFO is, primarily, seen as an ontology of types (universals) rather than particulars (individuals). Types extend to general classes of particulars which share important features. For example, the authors of this article are particulars that instantiate the type \emph{Human Being}; a red blood cell in the aorta of the first author and a hepatocyte in the liver of the last author are both particulars that instantiate the type \emph{Cell}. 
OWL ontologies are, however, ontologies of particulars and their classes. Types and classes look, \emph{prima facie}, isomorphic, and in practice they are often used as synonyms. However, there are subtle differences: 
classes can be straightforwardly defined via extensions of types: The class $A_{class}$ is defined as the cross-temporal aggregation of all entities that instatiate the 
type $A_{type}$. However, classes can also be defined using logical operations such as negation, resulting in classes like \emph{Non-human-animal}, which cannot seen as the extensions of a single type.  

In BFO, the realms of types and individuals (or classes and their members) 
are strictly disjoint: no individual can be instantiated, and no class can be a member of another class. The relation \mirel{instance\_of} links individuals to their types. A single instance may instantiate many different types (or analogously, be member of many different classes). 

In the remainder of this paper we will simplify our model to classes (independently of whether they are extensions of types) and individuals (which are members of classes).  

Fig.\ \ref{fig:bfoclass} shows the class and relation hierarchy of BFO 2.  


 
Whereas BFO 1.x was restricted to a tree of mutually disjoint classes, BFO 2.0 now offers an extensive set of relations. 
% Table~2 details these and others together with their definitions. 
Most of these relations relate individuals, not classes or types, which means that when used in class-level axioms, they are 
subject to an appropriate quantification. Axioms in BFO2 are mainly of the type domain and range restriction, i.e. they restrict the domain of the relation $\mirel{inheres\_in}$ to specifically 
dependent continuants, or the range of the relation $\mirel{participant\_of}$ to processes.  



\subsection*{Web Ontology Language (OWL)}

OWL (Web Ontology Language) is a family of ontology development languages, which were developed in the context of the Semantic Web, 
standardised by the the W3C, and which are supported by a wide range of tools including ontology editors such as Prot\'eg\'e 
and automated reasoners. OWL version 2 is the current one \cite{grau2008}, proposing several language profiles of different expressiveness. All but the profile called OWL full are based on Description Logics (DL), a family of
representation formalisms designed to build large scale ontologies \cite{baader2007dlhandbook}, all of which are decidable fragments of first-order logics. 
Thes allows classes and relations to be organized in hierarchies, and allows constraining axioms on classes to be formalized. Optionally, assertions about individuals can also be captured. 

OWL is increasingly being used in bio-ontologies such as the GO and ChEBI, and this in turn has been driving further biological applications. For example, aspects of OWL expressivity in ChEBI axioms has recently been used to implement an enhanced measure for class-class similarity \cite{ferreira2013exploiting}.

%OWL comprises several different ``profiles'', most of which correspond to DL dialects that are decidable subsets of first-order logics (FOL).

As discussed, proper representation of temporal quantification requires the use of ternary relations, with time as their third argument. While there has been work on expressive description logics which might underlie future OWL extensions that try to transcend this limitation \cite{Calvanese:1997} and also on description logics that explicitly account for temporality (e.g. \cite{Wolter:2001}), there is as yet no strong push towards standardisation of those formalisms, and tools suitable for end users are not readily available. For the remainder of this paper, therefore, we will work within the expressivity of the core DL profile of OWL 2 which is specified in \cite{OWL2:direct}.  

%\todo[inline, size=\small]{More on OWL syntax and semantics (mapping to FOL) for: classes (which
%are the extensions of BFO types or correspond to logical expressions formed by the extensions of BFO types, so-called defined classes)
%Individuals (the concrete entities that instantiate BFO types and / or are members of BFO classes). }
 
The semantics of OWL follows the principles of set theory. OWL object properties are binary predicates between OWL individuals. %They can optionally be connected by so-called property chains.
Well-formed class-level expression therefore require quantifiers when including object properties, such as $\forall$ (\texttt{only}) or $\exists$ (\texttt{some}).

%% ?$\mclass{Lung}\;\mathtt{subClassOf}\;\mirel{`has part'}\;\mathtt{some}\;\mclass{LobeOfLung}$''

For instance, the class expression $\mirel{continuant\_part\_of}\;\mathtt{some}\;\mclass{Cell}$ specifies the class that contains all individual things
that are necessarily part of at least one member of the class $\mclass{Cell}$.
The main axiom types available in OWL ontologies are $\mathtt{rdfs:subClassOf}$, which allow the construction of class subsumption hierarchies,
$\mathtt{owl:equivalentClass}$, with which class logical equivalences can be captured, and $\mathtt{rdf:Type}$, with which individuals can be assigned to the classes they are members of.
For example, $\mclass{CellMembrane}\;\mathtt{subClassOf}\;\mirel{continuant\_part\_of}\;\mathtt{some}\;\mclass{Cell}$
means that every member of the class $\mclass{CellMembrane}$ is
part of at least one member of the class $\mclass{Cell}$.
The big advantage of OWL ontologies is their formal rigor,
their well-studied computational behaviour, their rather
intuitive syntax (compared to first-order logics), the presence of editing tools like Prot\'eg\'e,
and their support by DL reasoners like HermiT and Fact++, which perform
reasoning tasks like consistency and satisfiability checking.
%, which
%  will be used in the evaluation.
%  instantiation, subclasses, object properties, quantification
%  How relations between classes are actually interpreted as relationships between individuals