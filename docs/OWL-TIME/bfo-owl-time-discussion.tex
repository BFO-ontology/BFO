%%%
%%% Nice and good evaluation of the different approaches goes here
%%%\todo[inline, size=\tiny]{many of the issues raised by Alan and Chris in the mailing list discussions}

%\section*{Survey of existing approaches}
In the following, we will provide a survey of existing approaches and discuss them in the light of our models. 

\subsection*{Histories in BFO 2}

BFO 2, which is currently available as an advanced draft \cite{BFO2:ref}, 
sketches out a more detailed theory of the
relationship between continuants and occurrents
%(SPAN objects)
they participate in. Specifically, BFO 2 makes the assumption that for each material object there
exists a special process, the \emph{history} of the object, which encompasses
``the totality of processes taking place in the spatiotemporal region occupied by
the entity.'' This means that there is a one to one correspondence between continuants
%(on the SNAP side)
and certain processes, which effectively
provides a ``bridge'' between a 3D and a 4D perspective. No complete formal theory
of histories, which have previously been described in \cite{cornucopia}, is available as of yet.


\subsection*{Conventional Modellers' Strategy for Temporalised Relations in OWL DL}

As explicit semantics for modelling temporal dynamics are not available in OWL
2, modellers tend to implicitly treat object properties as committing to a
``for all times'' interpretation in order to avoid obvious problems with the
entailed models. For instance, in an anatomy ontology like the Foundational Model of Anatomy (FMA), the object
property \mirel{has\_part} is transitive, and used in axioms such as
%
\begin{equation}
\begin{split}
\mclass{Lung}\;\mathtt{subClassOf}\;\mirel{has\_part}\;\mathtt{some}\;\mclass{LobeOfLung}
\end{split}
\label{eq:lung}
\end{equation}    
%
and
%
\begin{equation}
\begin{split}
\mclass{LobeOfLung}\;\mathtt{subClassOf}\;\mirel{has\_part}\;\mathtt{some}\;\mclass{BronchiopulmonarySegment}
\end{split}
\label{eq:lunglobe}
\end{equation}    
%
%
If the underlying interpretation were ``part for some time'', transitivity of the
binary \mirel{has\_part} could no longer be taken for granted.
%, as two
%\mirel{has\_part} assertions to be chained could belong to two different SNAP
%ontologies.
%\subsection*{Strategies Advanced in Related Work}
%Existing Design Patterns
%\subsubsection*{Default Reinterpretation}
%On the default reinterpretation of the OWL direct semantics to preserve the RO
% generic relatedness default reading.
%\todo[inline, size=\tiny]{I fused the default interpretation with the ``standard'' interpretation. Both mean the same IMHO}
%
% DIRECT SEMANTICS ALREADY INTRODUCED ABOVE
%Obviously, the interpretation must be equivalent to the OWL 2 direct semantics
%\cite{OWL2:direct} in as far as it preserves the syntactical structure and inferences and does not
%lead to additional expressivity. But it has � to our knowledge � never been made
%explicit what this substitution might consist of.  This is even more significant
%as this does not constitute at all a side issue or an idiosyncrasy of biomedical
%ontologies. On the contrary, virtually all OWL ontologies contain axioms on
%classes of continuants using binary object properties and leave the exact
%interpretation unexplained.

It is surprising that it has not been spotted earlier that virtually all OWL ontologies contain axioms on
classes of continuants using binary object properties and leave the exact interpretation re time unexplained.
In a similar vein RDF encoded triple data abound in which the temporal context is ignored, or, at best, derivable from the context.

To address this mismatch and try to understand it better we sketch here a
possible elucidation  which consists in a modification of the interpretation
function. We here make explicit what seems to be trivial for many OWL DL users, \emph{viz.}
that a DL formula is to be interpreted as the set of possible worlds over which the formula holds,
which would also include all temporal contexts.
%\todo[inline, size=\tiny]{Check this argument}.

The general strategy of this interpretation is to augment the
interpretations of class members and object properties in the OWL model with an
additional time index $t$ which specifies that the entity in question exists
(object property holds) at $t$. Class instances then become pairs and object
property instances triples. In order to keep the surface grammar and overall
semantics intact, the interpretations of all OWL axioms will be prepended with a
conditionalised universal quantification over $t$  hat specifies that the axiom
should hold at all times that the entity in question exists.  Time instants are
hereby external to the domain. For example, the interpretation of a class
assertion axiom that asserts that $a$ is an instance of class \mclass{C}, as long
as $a$ exists, would then read (domain~$\Delta$, interpretation~$\cdot^\mathcal{I}$):

\begin{equation}
\forall t:\;\pair{\dlint{a}}{t}\in \dlint{\Delta} \rightarrow
\pair{\dlint{a}}{t} \in \dlint{\mclass{C}}
\end{equation}

We implicitly assume that $\Delta$ contains individual/time-point pairs only for
those times at which an individual exists. Notably, this is only sufficient to express rigid instantiation: Whenever an individual exists at
all, it is also a member of the class it instantiates. The interpretation of
temporality-sensitive relations will become clear when we spell out the semantic
rules of existential quantification and value restriction, both of which assert
permanent generic relatedness because they apply existential quantification over the
object property range so that at each point in time a different individual of
class \mclass{B} can serve as a relatum. We will use the canonical structural syntax
\cite{OWL2:structural}
to ease comparison with the specified semantics \cite{OWL2:direct}.

\begin{description}
\item[Existential quantification ($\mrelb{rel}\;\mathtt{some}\;\mclass{B}$)]
\begin{equation}
\begin{split}
\dlint{\text{ObjectSomeValuesFrom}&(\mrelb{rel},\mclass{B})} =_{def}\\ &\quad
\{\pair{\dlint{a}}{t}\in \dlint{\Delta}\,|\; \exists b:\;\langle\dlint{a},b,t\rangle
\in \dlint{\mrelt{rel}} \wedge \pair{b}{t} \in \dlint{\mclass{B}}\}
\end{split}
\end{equation}
\item[Value restriction ($\mrelb{rel}\;\mathtt{only}\;\mclass{B}$)]
\begin{equation}
\begin{split}
\dlint{\text{ObjectAllValuesFrom}&(\mrelb{rel},\mclass{B})} =_{def}\\ &\quad
\{\pair{\dlint{a}}{t}\in \dlint{\Delta}\,|\; \forall b:\;\langle\dlint{a},b,t\rangle
\in \dlint{\mrelt{rel}} \rightarrow \pair{b}{t} \in \dlint{\mclass{B}}\}
\end{split}
\end{equation}
\end{description}

In OWL object property assertions the time index is bound through universal
quantification again:

\begin{equation}
\text{ObjectPropertyAssertion}(\mrelb{rel},a,b) =_{def}\;\forall
t:\;\pair{\dlint{a}}{t} \in \dlint{\Delta} \rightarrow \langle
\dlint{a},\dlint{b},t\rangle \in \dlint{\mrelt{rel}}
\end{equation}

Hence, object property assertions specify permanent relatedness. Disregarding
the difference between specific and generic permanent relatedness for the time
being, this interpretation of
OWL 2 at least successfully mimics the semantics of class level
relations intended by the relations ontology (RO, \cite{OBO:RO}) and allows us to think of
the syntactical forms represented in Table \ref{tab:syntaces} as equivalent.
\begin{table}
\caption{Syntactical representations of (permanent) relatedness expressions}
\label{tab:syntaces}
{
\begin{tabular}{p{10.9em}cp{10.5em}}
\toprule
\parskip=0cm
\parbox{10.9em}{\centering OBO Syntax} & OWL (Manchester Syntax) & \parbox{10.5em}{\centering First Order Logic} \\
\midrule
\texttt{$[$Term$]$}\par
\texttt{id:} \mclass{A}\par
\texttt{relationship:} \mrelb{rel} \mclass{B} &

\parbox[t][1.5em][c]{11.2em}{\centering $\mclass{A}\;\mathtt{subClassOf}\;\mrelb{rel}\mathtt{some}\mclass{B}$} &

$\forall a,t:\;\mrelt{inst}(A,a,t)\rightarrow$\par
$\quad \big(\exists b:\;\mrelt{inst}(B,b,t)\;\wedge$\par
$\qquad\;\mrelt{rel}(a,b,t)\big)$\\
\bottomrule
\end{tabular}
}
\end{table}

This approach  also retains standard transitivity semantics of OWL 2 object
properties, so that quantification over time maintains transitivity of the
relation in question.
This can be shown, e.g. for the transitive relation \mirel{has\_part}: if an organism has
some heart at any time, and if this heart has some heart valve at any time, then
the organism has some heart valve at any time:

\begin{prooftree}
\AxiomC{$\mclass{A}\;\mathtt{subClassOf}\;\mrelb{has\_part}\;\mathtt{some}\;\mclass{B}$}
\AxiomC{$\mclass{B}\;\mathtt{subClassOf}\;\mrelb{has\_part}\;\mathtt{some}\;\mclass{C}$}
\BinaryInfC{$\mclass{A}\;\mathtt{subClassOf}\;\mrelb{has\_part}\;\mathtt{some}\;\mclass{C}$}
\end{prooftree}
And while there is nothing to be gained by actually modifying OWL 2 to use this
interpretation, it is very important that ontology engineers are aware of the
implications of their modelling decisions with regard to relations that are
sensitive to the issue of temporal strength. However, this approach
still has the consequence that ``\emph{temporary relatedness}'' cannot be
expressed directly in an OWL 2 ontology, so we need to look for more involved
solutions to the problem.

\subsection*{Reification}

A common strategy to work around the limitations of description logics is to
represent ternary relations through reification. Reification involves the
introduction of a class $\mclass{C_\mrelt{rel}}$ for each ternary relation
\mrelt{rel}. The relata of \mrelt{rel}
are then connected to instances of $\mclass{C_\mrelt{rel}}$ by three new binary
relations \mrelb{R_1}, \mrelb{R_2},
\mrelb{R_3}. The instance-level assertion
$$
\mrelt{rel}(a,b,t)
$$

would then be transformed into the following statement:\todo[inline, size=\tiny]{Changed from list of statements to proper axiom}
\begin{equation}
\exists x: \mclass{C_\mrelt{rel}}(x) \wedge
\mrelb{R_1}(x,a) \wedge
\mrelb{R_2}(x,b) \wedge
\mrelb{R_3}(x,t)
\end{equation}

Such proposals have, with a varying degree of sophistication, seen quite a bit
of dissemination in the ontology engineering community \cite{ODP:nary}, but they suffer
from unavoidable drawbacks. Most obviously, they are rather complex. This bears
the risk of errors in the ontology engineering process and decreases reasoning
efficiency \cite{Grewe:2010}. To address the complexity problem, it has been suggested to
select reification classes based on what seems ontologically ``fitting'' for the
domain of an ontology \cite{Fiadeiro:2010}.
\todo[inline, size=\tiny]{unclear how transitive relations can be expressed by this without proliferation of object properties}

\subsection*{Welty/Fikes: Fluents}
The prototypical approach for dealing with temporally changing information in
OWL within a four-dimensionalist framework was provided by
\cite{Welty:2006}. And while the proponents  agree that the 4D  approach is ``clearly
not something that immediately appeals to common sense'', they also claim
that it ``gives us another tool to use when solving a practical problem.'' To this
end, they present an ontology that models fluents, i.e. ``relations that hold
within certain time interval but not in others.'' This works by considering all
entities as four dimensional entities that have temporal parts (time slices),
such that the material object property assertions hold (synchronously) between
time slices. For example, temporary relatedness could be expressed as in
(\ref{eq:leaf}).
%
\begin{equation}
\mclass{Leaf}\;\mathtt{subClassOf}\;\mathtt{inverseOf}(\mrelb{time\_slice\_of})\;\mathtt{some}\;(\mrelb{has\_part}\;\mathtt{some}\;(\mrelb{time\_slice\_of}\;\mathtt{some}\;\mclass{Tree}))
\label{eq:leaf}
\end{equation}
%
This is by far one of the most straightforward translations of the
four-dimensionalist commitment, but it suffers from considerable verbosity. It
gets even worse if permanent relatedness is concerned. In this case, the
above expression would have to be amended to include an
$\mathtt{inverseOf}(\mrelb{time\_slice\_of})\;\mathtt{only}$ clause to ensure that
all time slices of the entity are appropriately related to a time slice of the
other entity.

\subsection*{Zamborlini/Guizzardi: Moments, Relators and Qua-Individuals}
The commitment that some relational expressions are in fact better accounted for
as proper entities is also prominent in Zamborlini and Guizzardi's treatment of
contingent properties \cite{Zamborlini:Guizzardi}. For them, certain ``material
relations'' only hold by virtue of a separate truthmaker, the so called
\emph{relator}, which is formed by combining the ``qua individuals'' that
partake in the relation. Qua individuals abstract away certain aspects of an
individual so that only that information remains which is relevant for the
individuals participation in the relation.

Both kinds of entities are examples of ``moments'' in their
nomenclature, which are said to inhere in individual entities and can thus be
compared to dependent continuants or occurrents (respectively) in BFO
parlance. But while relators might often be appropriately represented by BFO
processes, the admissibility of qua individuals into BFO might be questionable
since they can hardly be aligned with BFO's realist commitment (where sheer
abstractions could only be regarded as artefacts of a persons though process).

Underneath the level of qua individuals (e.g. ``$\mclass{LeafQuaPartOfTree}$'')
and relators, there is the assumption of an ontology of time slices not unlike
the one in \cite{Welty:2006}, such that
temporal overlap between the qua individuals related by the relator can be
enforced.

Zamborlini and Guizzardi cite as an advantage for this approach that it is capable
of representing the persistence of a relationship across multiple time slices
without mentioning each explicitly (because the relator is associated with the
qua-individual and not its time slice). This is part of a set of requirements
suggested for modelling temporally changing information:

\begin{enumerate}
\item Avoid duplication of the other time slices if one entity partaking in the
relation changes.
\item Provide a consistent ontological interpretation of contingent (non-rigid)
instantiation.
\item Avoid repeating persisting properties for each time slice
\item Ensure that immutable properties of an entity cannot be overridden by a
time slice.
\end{enumerate}

We believe these points to be a good starting point for the evaluation of
any proposal to address the problem of time-dependent relation and should be
used to supplement our initial requirements.

\subsection*{Gangemi: Descriptions and Situations}
\todo[inline, size=\tiny]{should specify what is meant by situations here}
Aldo Gangemi's DnS pattern \cite{Gangemi:DnS} deserves mention because it treats
time-dependence of relations as a special case of perspectivity which can be
accounted for by the very heavy-duty reification mechanism of descriptions and
situations. In this case, the suggestion is to use the situation pattern in
order to associate the relata and their temporal context with a common
situation, which is effectively a reified assertion (a proposition). Again, such
entities are figments of the mind and can only be admitted into a realist
ontology such as BFO as such -- rather than being a general way to refer to
arbitrary facts.

Notably, though, Gangemi reminds us of the fact that OWL 2's $\mathtt{hasKey}$
axiom can be used circumvent the problem of possible duplication of instances
for the same relational $n$-tuple: If a situation $\mclass{S}$ \todo[inline, size=\tiny]{don't understand}
were to use the properties
$\mrelb{has\_timeStamp}$, $\mrelb{has\_subject}$, and $\mrelb{has\_object}$, the axiom
\begin{equation}
\mclass{S}\;\mathtt{hasKey}(\mrelb{has\_timeStamp}, \mrelb{has\_subject},
\mrelb{has\_object})
\end{equation}
would ensure that duplicate entities would be coalesced in the model.


\subsection*{Temporal extension to Description Logics}

Extensions to description logics to include temporal notions have been proposed by ... 

We have not considered these approaches because they are not covered by the OWL specification and 
do not yet provide mature tools like editors and reasoners. 

\subsection*{Performance}
The benchmark results show no great differences between TR and TQC, 
however with quite lower classification times for Fact++ compared with HermiT. 
If adding examples the decrease in performance is more accentuated for TQC compared to TR, 
which is explained by the lower number of A-Box axioms in the latter, due to the 
impossibility of representing specific times. 

We also observed that declaring the has\_max relation as functional has a negative
impact on the Fact++ reasoner, which fails to terminate, while 
HermiT does terminate. (Check). Setting has\_max and max\_of to be reflexive 
causes inconsistencies (explanations hint at an interference with existsAt). 

\subsection*{Relevance (or not) to other top-level ontologies}
\todo{todo[inline, size=\tiny]{do we need this subsection?}}
%%However, the proposal sketched here could potentially be implemented in other top-level
%%ontologies that are sufficiently similar to BFO, including DOLCE \cite{DOLCE:ref} and \cite{GFO:ref}.
% I don�t think GFO has quite the same occurrent-continuant distinction 


\subsection*{Patterns}

From a usability point of view, TR may have an advantage due to the embedding
of a complex meaning within relations. While the higher number of relations might 
complicate navigation in relation hierarchies, the axioms built out of them appear 
more straightforward. However, this requires that users completely understand the 
meaning of each relation. This is a challenge especially with the inverse relations, 
and the unintuitive fact that, e.g. `has part at all times' is not the inverse of 
`part of at all times'.

TQR is the leaner ontology, due to a much more concise set of relations. However, 
it requires a fundamental understanding of the meaning of TQCs. Syntactically,  
most axioms look the same as in the modelling strategies that ignore temporal qualification.
This might be a advantage for the adaption of existing domain ontologies.
The expression of ``some time'' relatedness requires an additional nesting, using the relation
``at\_some\_time'', which connects temporal instances of the same continuant. 
In particular, four patterns can be distinguished:

\begin{itemize}
\item
Class-level assertion $A$ valid for a given TQC individual $c$:
%
\begin{equation}
\begin{split}
\minst{c}\;\mathtt{Type}\;\mclass{A}
\end{split}
\end{equation}    
%
\item
Class-level assertion $A$ valid for all related temporal slices of a given TQC individual $c$:
%
\begin{equation}
\begin{split}
\mirel{at\_some\_time}\;\mathtt{value}\;\minst{c}\;\mathtt{subClassOf}\;\mclass{A}
\end{split}
\end{equation}    
%
\item
Class-level assertion $A$ valid for at least one temporal slice for each member of a TQC class $C$:
%
\begin{equation}
\begin{split}
\mclass{C}\;\mathtt{subClassOf}\;\mirel{at\_some\_time}\;\mathtt{some}\;\mclass{A}
\end{split}
\end{equation}    
%
\item
Class-level assertion $A$ valid for all related temporal slices of all members of a TQC class $C$:
%
\begin{equation}
\begin{split}
\mclass{C}\;\mathtt{subClassOf}\;\mclass{A}
\end{split}
\end{equation}    
%
\end{itemize}
