\section*{Representing Time}


\subsection*{\\
The underspecification of time representations in OWL axioms}

Both the OBO axiom
%
\begin{equation}
\begin{split}
\mclass{CellMembrane}\;\mirel{continuant\_part\_of}\;\mclass{Cell}
\end{split}
\label{eq:cellobo}
\end{equation}
%
and the OWL axiom (often considered equivalent)
%
\begin{equation}
\begin{split}
\mclass{CellMembrane}\;\mathtt{subClassOf}\;\mirel{continuant\_part\_of}\;\mathtt{some}\;\mclass{Cell}
\end{split}
\label{eq:celldl}
\end{equation}
%
do not exhibit any explicit commitment with regard to time. This is not a side aspect, because it makes a
difference whether, e.g., a cell membrane is \emph{always} part of some cell or
only \emph{at some time}. 
The lack of temporal definition of OWL statements is especially unsatisfactory
when it comes to transitive properties like $\mirel{continuant\_part\_of}$ or $\mirel{located\_in}$,
where the suppression of the temporal factor can produce plainly wrong entailments,
especially at the level of individuals, such as in the following example:
%
\begin{equation}
\begin{split}
\mirel{located\_in} (\mirel{Thrombus\#39874}, \mirel{Heart\#431234})  \\
\mirel{located\_in} (\mirel{Heart\#431234}, \mirel{Patient\#900812})
\end{split}
\label{eq:trans}
\end{equation}
%
Assuming that the thrombus was no longer in the heart when it was transplanted to Patient\#115678, then the entailment
%
\begin{equation}
\begin{split}
\mirel{located\_in} (\mirel{Thrombus\#39874}, \mirel{Patient\#115678})
\end{split}
\label{eq:transEntailment}
\end{equation}
%
which follows from the transitivity property of the relation $\mirel{located\_in}$, is obviously wrong.

This has a far-reaching consequence: by default, OWL statements of this form are temporally ambiguous in a way that may entail unintended reasoning consequences when domain ontologies are used with real data. (A lack of an explicit treatment of temporal context has previously been implicated in an assessment of the quality of existential restrictions in OBO Foundry ontologies \cite{boeker2011}.)

\subsection*{Strengths of Relatedness}

We will propose a distinction between different temporal \emph{strengths} of relatedness. Before, however, we need to further characterise the ontological status of time, as referred to by the symbol $t$
in our further deliberations. We refrain from a distinction between time intervals and time points, as the latter can be approximated by infinitesimally small intervals.
We restrict ourselves to those ternary relations $\mrelt{rel}$ for which it can be assumed that whenever they hold for a time interval, 
they also hold for any of its subintervals, including time points. E.g., if a car is located in a garage during a whole day, it is located therein at any moment or time interval within this day. If a patient's cholesterol level is elevated during a whole year, it is elevated at any shorter amount of time during this year. 

Formally:
\begin{equation}
\begin{split}
\forall a, b, t, t^\prime:\; \mrelt{rel}(a, b, t) \wedge \mirel{within}(t^\prime,t) \rightarrow
\mrelt{rel}(a,b,t^\prime)
\end{split}
\label{eq:temporarily:temp}
\end{equation}

\subsubsection*{Some Time Relatedness (STR)}

Informally: for all a instances of \mclass{A} there is some time $t$ and some instance $b$ of
\mclass{B} such that $a$ is related to $b$ at $t$. Examples:
\begin{enumerate}[(a)]
\item for all apple seeds there is
some apple such that the seed is part of the apple at some time.
\item for all
trees there is some leaf such that the leaf is part of the tree at some time.
\end{enumerate}

Formally:
\begin{equation}
\begin{split}
\mclass{TemporarilyRelated}(\mclass{A},\mclass{B}) =_{def}&\;
\forall a, t:\; \mrelt{inst}(\mclass{A}, a, t) \\
&\ \rightarrow
\exists b, t^\prime:\;(\mrelt{inst}(\mclass{B},b,t^\prime) \wedge
\mrelt{rel}(a,b,t^\prime) \wedge \mirel{within}(t^\prime,t))
\end{split}
\label{eq:temporarily:cls}
\end{equation}

\subsubsection*{Permanent Generic Relatedness (PGR)}

Informally: for all instances $a$ of \mclass{A} there is, at all times $t$ that
$a$ exists,
some instance $b$ of \mclass{B} such that $a$ is related to $b$ at $t$, but not necessarily
always the same $b$ at all times $t$. Examples:
\begin{enumerate}[(a)]
\item all cells have a water molecule as
part at all times, but not always the same water molecule.
\item every bacteria colony has some bacteria as parts at all times, but not
always the same bacteria.
\end{enumerate}

\begin{equation}
\begin{split}
\mclass{PermanentlyGenericallyRelated}(\mclass{A},\mclass{B}) =_{def}&\;
\forall a, t:\; \mrelt{inst}(\mclass{A}, a, t) \\
&\ \rightarrow
\exists b:\;(\mrelt{inst}(\mclass{B},b,t) \wedge
\mrelt{rel}(a,b,t))
\end{split}
\label{eq:generically:cls}
\end{equation}

This case is the standard interpretation of OBO binary class-level relations such as in formula (\ref{eq:cellobo}), according to \cite{OBO:RO}.

\subsubsection*{Permanent Specific Relatedness (PSR)}

Informally, for all instances $a$ of \mclass{A} there is, at all times $t$ that $a$ exists, an
instance $b$ of \mclass{B} such that $a$ is related to $b$ at $t$; in this case it is always the
same $b$ at all times $t$. Examples:
\begin{enumerate}[(a)]
\item a human being has a brain as part at all times, and it is necessarily the same brain.
\item a radioactively marked molecule of DNA has the radioactive isotope as part
at all times, and it is necessarily the same atom.
\end{enumerate}

\begin{equation}
\begin{split}
\mclass{Permanently}&\mclass{SpecificallyRelated}(\mclass{A},\mclass{B}) =_{def}\;
\forall a, t:\; \mrelt{inst}(\mclass{A}, a, t) \\
&\ \rightarrow
\exists b:\;\big(\mrelt{inst}(\mclass{B},b,t) \wedge
\mrelt{rel}(a,b,t))
\\
&\quad\quad \wedge \forall t^\prime: (\mrelt{inst}(\mclass{A},a,t^\prime)
\rightarrow (\mrelt{rel}(a,b,t^\prime) \wedge
\mrelt{inst}(\mclass{B},b,t^\prime))\big)
\end{split}
\label{eq:specifically:cls}
\end{equation}

