%%%
%%% RESULTS
%%%
\subsection*{OWL models}

% 23 cqs - TR 9 concording - TQC 6 concording - TR 5 incomplete models 


According to the specifications two OWL models for BFO were used. 
The first one, implementing the TR (temporalized relations) approach, is a minor update to the BFO2OWL ``Graz'' release 
(July 2012), with 78 object properties. 
The second one was newly built following the BFO2 specification and extending the set of object properties 
only by ``\textbf{at some time}'', ``\textbf{has max}'' and ``\textbf{spans}'', with a total number of 58 object properties. 

The results of the OWL implementations of the examples can be inspected in the additional files. 
Table \ref{tab:results} provides an overview of the results. Each of the 23 examples had been implemented in both TQR and TR. In TR
the models for STR\_im+, STR\_im-, PGR\_im-, NR\_t and NR\_i were underspecified, 
as the TR implementation does not address specific time intervals. 

Of those competency questions for which satisfiability was expected (n= 5), the TR model 
was satisfiable in one case, whereas the TQC model was satisfiable in all cases.  

Of the competency questions for which insatisfiability was expected (n= 18), 
this could be confirmed by the TR model in seven cases, whereas it was obtained by the TQC model in eleven cases. 

\begin{table}
\caption{Overview of results of competency question. Comparison between TR and TQC.  The asterisk 
marks those examples which could con be fully modelled.}
\label{tab:results}
{
\begin{tabular}{p{10.9em}cp{10.5em}}
\parskip=0cm
\parbox{10.9em}{(see separate PDF file)} \\
\end{tabular}
}
\end{table}






In detail the following observations are considered noteworthy for the TR approach:

\begin{itemize}
\item
As the TR models of some time relatedness do not make their semantics explicit, an important
class of modeling patterns are too strong. Generally spoken, if $A$ is ``all time related'' to $B$,
it is also ``some time related''. For instance, we want to say that although every anatomical structure 
is part of an organism at least during some time, there are anatomical structures that exist even severed 
from the body, e.g. teeth. In TR this produces a contradiction, due to the abovementioned reason, whereas 
in TQC the statement there are temporal slices (TQCs) in which the parthood predicate does not hold perfectly   
harmonizes with the statement that it holds with other TQCs.

\item
As TR does not provide sufficient resources that would allow for stating a relatedness during a 
certain time, important inferences such as that an object cannot be at a place p' at a given time 
because it is already at p (which does not share parts with p') cannot be drawn. However, TQR here
reaches its limits if the reasoning has to account for different, possibly overlapping time intervals. 
An example is our example with the swollen ankle. Both approaches fail to demonstrate the contradiction 
in the statement that the ankle is swollen during a given time period but is not swollen during an interval
within that period.  

\todo{find related work, e.g. re Allan's interval calculus and DL}

\item
As a consequence of the previous observation, the representation of phased sortals, here 
the embryo / fetus example, is incomplete and leads to inconsistencies in case the phased sortal classes 
are introduced as disjoint. A distinction between (i) $c$ being a member of $C_1$ at one time and 
$C_2$ at a different time, and (ii) $c$ being a member of the intersection of the classes $C_1$ 
and $C_2$ is not possible.


\item
As generic all time relatedness can only be expressed in TR for mereological relations, we would 
expect a wrong result in the example with a generically dependent continuant, \emph{viz} 
the electronic health record. What happens is that in PGR\_cm- the TR model produces a correct
result, however based on an incomplete premise, \emph{viz.} stating that the data are on a storage 
device only for some time (which allows for the unintended model that they are not stored anywhere
for some time). 

\item
As expected, TR fares better where the specific all time relatedness. The advantage is exemplified
in the PSR\_cm+ reasoning example, where the (wrong) model in which a human gets a brain transplant
is rejected. However, both TR and TQC approaches fail with the example
of changing sex in mammals.
  
\end{itemize}
  
Classification times were measured for the DL classifiers Fact++ and HermiT 1.3.8. using an 8 core lenovo Think Pad T530 laptop with 
8 GByte memory, cr. \ref{tab:perf}.


\begin{table}
\caption{Classification times of ontologies in ms}
\label{tab:perf}
\centering
\begin{tabular}{ l c c c c }
\hline
 &  {\bf TR} & {\bf TQC} & {\bf TR + examples} & {\bf TQC + examples} \\
Fact++	 & 217	& 185	& 1,008 & 2,050 \\
HermiT & 2,580 & 1,736 & 41,801 & 38,665\\ 
\end{tabular}
\label{tab:cases}
\end{table}


  
  
