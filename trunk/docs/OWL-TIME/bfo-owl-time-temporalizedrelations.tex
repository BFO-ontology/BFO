
Of the three types of relational patterns, \emph{viz.} \emph{Temporary Generic Relatedness} (TGR), \emph{permanent generic relatedness} (PGR), and \emph{permanent specific relatedness} (PSR), TGR and PSR have in common that they can be asserted between individuals: a token $a$ can be temporarily related to a token $b$, e.g. 
a specific leaf is temporarily part of a specific plant, or a specific human, e.g. Barack Obama, 
is temporarily located at a specific place, e.g. insite Air Force One.

In a similar way, two individuals can be permanently related, for example via relations of structural dependence or essential parthood as in 
your body mass being permanently inherent in your body, or the sun being permanently part of our solar system.

This means that in all such cases  we can hide the time argument within a so-called temporalized relation. We will in the following call this approach ``binarization''. 
For TGR the corresponding binarized (superscript $^b$) derivates of ternary (superscript $^t$) relations are introduced as follows:  

\begin{equation}
\begin{split}
\mrelb{rel\_at\_some\_time} (a,b) =_{def}&\; \exists t: \mrelt{rel}(a,b,t)  
\end{split}
\label{eq:temporarily:ind}
\end{equation}

This relation schema can be used in FOL expressions such as the individual-level expression

\begin{equation}
\begin{split}
\mrelb{located\_in\_at\_some\_time} \;(\mathrm{BarackObama}, \mathrm{AirForceOne})  
\end{split}
\label{eq:personlocated}
\end{equation}
%
or in the class-level axiom
%
\begin{equation}
\begin{split}
\forall a:\; \mrelb{inst}(\mclass{Leaf}, a) 
\rightarrow
\exists b:\;\mrelb{inst}(\mclass{Tree}, b)
\wedge
\mrelb{part\_of\_at\_some\_time}(a,b)  \end{split}
\label{eq:leaf2}
\end{equation}
%
with the binary instantiation relation $\mrelb{inst}$:
%  
\begin{equation}
\begin{split}
\mrelb{inst} (\mclass{A},a) =_{def}&\; \forall t: \mrelt{inst}(\mclass{A},a,t)  
\end{split}
\label{eq:temporarily:inst}
\end{equation}
%
Analogously, the PSR pattern is introduced as follows:  
%
\begin{equation}
\begin{split}
\mrelb{rel\_at\_all\_times} (a,b) =_{def}&\;
\forall t: \; (\mrelb{exists\_at} (a, t) \; \rightarrow \; \mrelb{exists\_at} (b, t) \; \wedge \; \mrelt{rel}(a,b,t))  
\end{split}
\label{eq:permanently:ind}
\end{equation}
%
This relation can be used in an FOL assertion such as:
%
\begin{equation}
\begin{split}
\mrelb{located\_in\_at\_all\_times} \;(\mathrm{Sun}, \mathrm{Solar\_System})  
\end{split}
\label{eq:earth}
\end{equation}
%
or in the class-level axiom
%
\begin{equation}
\begin{split}
\forall a:\; \mrelb{inst}(\mclass{Vertebrate}, a) 
\rightarrow
\exists b:\;\mrelb{inst}(\mclass{Spine}, b)
\wedge
\mrelb{has\_part\_at\_all\_times}(a,b)  \end{split}
\label{eq:spine}
\end{equation}
%  
Different sorts of temporalized relations behave differently as concerns properties such as transitivity.
In case a ternary relation is transitive, i.e.  
%
\begin{equation}
\begin{split}
\mrelt{rel}(a,b,t) \; \wedge \; \mrelt{rel}(b,c,t) \rightarrow \; \mrelt{rel}(a,c,t)   
\end{split}
\label{eq:trtrans}
\end{equation}    
%
the derived binarized relation is not transitive, if the underlying relation TGR (cf. Formula \ref{eq:temporarily:ind}) is 
not necessarily identical for the conjoints in a transitive chain. If Obama is in Air Force One at some time, 
and Air Force One is in Oklahoma City Air Logistics Complex (OC-ALC) at some time, this does not imply that Obama 
was ever at OC-ALC. For the property of having an inverse relation, however, TGR relations face no problem.  
% 
\begin{equation}
\begin{split}
\mrelt{rel}(a,b,t) \; = \; \mrelt{inv\_rel}(b,a,t)  
\end{split}
\label{eq:trinv}
\end{equation}    
%
trivially entails
%
\begin{equation}
\begin{split}
\mrelb{rel}(a,b) \; = \; \mrelb{inv\_rel}(b,a)  
\end{split}
\label{eq:trinv2}
\end{equation}    
%
If Obama is contained in Air Force One at some time, then Air Force One is the container of Obama at some time.

Matetrs are different with PSR relations. Here, it can be shown that transitivity is maintained, because by \ref{eq:permanently:ind} each such relation holds, 
for all times at which the first argument exists, this is also the case for the second argument, which thereby becomes the first argument of the next link in a transitivity chain.
Binatry PSR relations, however, face problems when it comes to the existence of inverses. 
Again from (\ref{eq:permanently:ind}), we see that the scope of the quantification over time is confined to the first argument. This entails the existence of the second argument only at times when the first argument exists. This however, does not preclude that the second might outlive the first. If we say that a certain vertebrate organism always has a spine, this does not preclude that the spine may still exist centuries after the animal's death.    
This means that the inverse of $\mrelb{has\_continuant\_part\_at\_some\_time}$ is not $\mrelb{part\_of\_continuant\_at\_some\_time}$ \todo{}, but something like $\mrelb{part\_of\_continuant\_at\_some\_time\_at\_which\_the\_whole\_exists}$.  

But it is the PGR case which is the standard interpretation of class level 
relations in OBO-RO \cite{OBO:RO}, and thus it is PGR that has influenced the development of biomedical ontologies 
for nearly a decade. On closer scrutiny, we can now see that some instances of OBO 
relational assertions correspond not to PGR but to TGR, for example in the Foundational Model of Anatomy 
every human heart is asserted to be part of some human organism. Under real-world conditions the heart may still continue to exist 
after the death of the organism (undergoing preparation for transplantation). 
Every human may have had teeth, but toothless humans are still humans and teeth can outlive their bearers. 
In other cases, PGR relations may be specialized as PSR, especially in the case of dependent continuants: 
the same redness of a red blood cell inheres in the cell as long as the cell exists. 
Or the surface of your body or the headache in your head cannot migrate to 
a different body. Nevertheless there are enough cases in which PGR is the only acceptable interpretation, 
especially if we consider PSR as a specialization of PGR.

A major drawback for the temporalized relations approach is the fact that a PGR relation cannot be expressed 
by means of a relation between individual entities. For instance, the statement ``every cell 
nucleus is part of some cell at all times at which the nucleus exists'' does not mean that a certain cell nucleus is part of a 
certain cell. It may become part of a different cell if the original cell fuses with another one. 
It is therefore not possible to assert PGR relations at the level of individuals. 
Therefore, the OBO standard case cannot be directly expressed using a temporalized relation.

However, there is a way to use temporalized relations to represent at least part of what is 
involved when we use relations for which PGR would be the most correct temporal strength.

The reasoning is the following: If we want to express that classes $A$ and $B$ are related by a PGR 
relationship, such as $A$ obo:$\mclass{located\_in}$ $B$\todo{BS: Give a real example}, we may resort to 
\emph{histories}, which are occurrents and can therefore be related by binary relations. According 
to BFO 2, a history is a complete process that is the sum of the totality of processes taking place 
in the spatiotemporal region occupied by a material entity or site. Thus, the history of a continuant 
$c$ can be seen as a four-dimentional spacetime worm with a temporal extension that equals the 
timespan $t$ during which $c$ exists and a series of three-dimensional spatial extensions each 
of which is indexed indexed by some point 
$t_i \in t$. 

Histories can be related by occurrent parthood relations, as in every phase (temporal part) 
of the history of a cell nucleus is part of the history of some cell (not necessarily of the same cell).  

Continuants and their histories are related by the $\mrelb{has\_history}$ relation 
(inverse: $\mrelb{history\_of})$.

If we want to express that all instances of the class $A$ are always located in some instance of the 
class $B$ we can express this in FOL by using the $\mrelb{temporal\_part\_of}$ relation, which holds 
between two occurrents when the former is a phase or subprocess (a slice or 
segment) of the latter, as contrasted with $\mrelb{occurrent\_part\_of}$, which is the general inclusion relation between two 
occurrents. Fig.\ \ref{fig:trgraph} visualizes a situation, in which every member of the class $A$ 
-- for instance $a$ -- is always located in some member of $B$, here represented by the 
four objects $b_1$ to $b_4$. This is expressed by the fact that any temporal part of the history of $a$ 
is part of the history of some member of $B$. 


Formally, with $h_a$ representing the history of $a$ :

\begin{equation}
\begin{split}
\mclass{Permanently} & \mclass{GenericallyLocatedIn}(\mclass{A},\mclass{B})  =_{def}  \\
& \forall a, h_a, h_a', t: \mrelt{inst}(\mclass{A}, a, t) \wedge \mrelb{inst}(\mclass{Occurrent},h_a') \\
& \wedge \mrelb{inst}(\mclass{Occurrent},h_a') \wedge \mrelb{temporal\_part\_of}(h_a', h_a) \wedge \mrelb{history\_of}(h_a, a) \rightarrow \\
& \quad \quad \quad \exists b, h_b: \mrelt{inst}(\mclass{B}, b, t) \wedge \mrelb{inst}(\mclass{History},j) \wedge \\
& \quad \quad \quad \mrelb{history\_of}(h_b, b) \wedge  \mrelb{occurrent\_part\_of}(h_a', h_b) 
\end{split}
\label{eq:history:PartOf}
\end{equation}



%\mrelb{inst}(\mclass{Occurrent}, o) \mrelb{inst}(\mclass{Occurrent}, j_p) \wedge \\ 
%& \quad \mrelb{inst}(\mclass{History},j) \wedge 
%  \wedge  
%\mrelb{occurrent\_part\_of}(o, h_p) \wedge \\
%& \quad \mrelb{occurrent\_part\_of}(h_p, h) \wedge
%\mrelb{historyOf}(h, b)

%
%
%
%
%\begin{equation}
%\begin{split}
%\mirel{temporalPartOf}\;\mathtt{some}\;
%(\mirel{historyOf}\;\mathtt{some}\;\mclass{A}) %\;\mathtt{subClassOf}\ \; \; \\
%\; \; \mirel{occurrent\_part\_of}\;\mathtt{some}\;%(\mirel{temporalPartOf}\;\mathtt{some}\;
%\;(\mirel{historyOf}\;\mathtt{some}\;\mclass{B}))
%\end{split}
%\label{eq:history:PartOf}
%\end{equation}    

%In a similar way, the converse case can be expressed:
%
%\begin{equation}
%\begin{split}
%\mirel{temporalPartOf}\;\mathtt{some}\;%(\mirel{historyOf}\;\mathtt{some}\;\mclass{C}) %\;\mathtt{subClassOf}\ \; \; \\
%\; \; \mirel{hasOccurrentPart}\;\mathtt{some}\;%(\mirel{temporalPartOf}\;\mathtt{some}\;
%(\mirel{historyOf}\;\mathtt{some}\;\mclass{D}))
%\end{split}
%\label{eq:history:has\_part}
%\end{equation}    
%
%In the case of cell nuclei and cells their generic parthood could be formalized as follows:

%\begin{equation}
%\begin{split}
%\mirel{temporalPartOf}\;\mathtt{some}\;%(\mirel{historyOf}\;\mathtt{some}\;\mclass{CellNucleus}) \;\mathtt{subClassOf}\ \; \; \\
%\; \; \mirel{occurrent\_part\_of}\;\mathtt{some}\;%(\mirel{temporalPartOf}\;\mathtt{some}\;
%\;(\mirel{historyOf}\;\mathtt{some}\;\mclass{Cell}))
%\end{split}
%\label{eq:historyCell}
%\end{equation}    

%Transitivity would be maintained, e.g. every nucleolus %is part of some cell because every nucleolus is part %of some cell nucleolus  

%\begin{equation}
%\begin{split}
%\mirel{temporalPartOf}\;\mathtt{some}\;%(\mirel{historyOf}\;\mathtt{some}\;\mclass{Nucleous}) \;\mathtt{subClassOf}\ \; \; \\
%\; \; \mirel{occurrent\_part\_of}\;\mathtt{some}\;(\mirel{temporalPartOf}\;\mathtt{some}\;
%\;(\mirel{historyOf}\;\mathtt{some}\;\mclass{CellNucleus}))
%\end{split}
%\label{eq:historyCellNucleus}
%\end{equation}    
%
%as it can be trivially shown.

However, the inclusion of phases of histories does not allow a distinction between location and parthood \cite{Jansen:Schulz}, so that the relation between a pregnant organism and an embryo or foetus would not be distinguished between the relation between an organism and its brain. \todo{BS: this could be fixed by taking sites into account. steschu: Example? }


\todo[inline, size=\tiny]{This section should be revised by Alan. Especially the following issues should be discussed: (i) PGR for continuant - continuant relations which are not mereological, such as inherence, concretization, (ii) application to process participants}
