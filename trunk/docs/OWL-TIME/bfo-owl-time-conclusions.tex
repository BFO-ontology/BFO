
In this paper we have addressed an inhering limitation of current OWL dialects,
\emph{viz.} their limitation to binary relations, whereas the current version 
of the Basic Formal Ontology, BFO2, requires ternary relations wherever
continuants are involved. The argument is that relations between continuants 
(such as spatial location) as well as between a continuant and an occurrent 
(e.g. participation of an object in a process) are time dependent. Therefore, 
such relationships require time as a third parameter. 
Up to now, builders of OWL ontologies have tacitly assumed relational expressions
in OWL to range over all instants in time. This assumption falls short of 
expressing relationships that do not hold for all times the relata exists, 
which entails improper and underspecified expressions both an A-box and 
a T-Box level. 

Two principally different approaches have been introduced, \emph{viz.} TR 
(temporalized relations), which embeds temporalization into OWL object properties, 
and TQC (temporally qualified continuants), which assumes all continuant instances
to be temporally qualified, i.e. referred to in the context of a time point or 
interval. Whereas both approaches manage quite well to distinguish between 
temporary and permanent relatedness, they fundamentally differ in their account 
of the latter. Here, TQC offers a straightforward solution for what we call 
generic relatedness, which focuses of the type of the relata rather than on 
the individual permanence of relationship. In contrast, TR prefers 
permanent relatedness and only offers an indirect approach to represent 
generic relatedness, which is furthermore limited to the spatial inclusion 
relation (expressed as the spatiotemporal inclusion relation between the 
\emph{histories} or the related continuants.

A series of 23 modelling examples was created, each of which represented a different
type of modelling problems. Each of them was challenged by one competency question, 
expressed as DL queries for class-level examples, and A-level expressions in case
of individual-level examples. They were tested for satisfiability and compared
to the theoretically established reference standard. 

As a result, concordance with the reference standard was achieved for 16 of 23 
representations using the TQC approach, and 8 of 23 using TR. The lower result for 
TR is affected by its limited expressiveness regarding time-indexed relationships
at an A-box level. Since the reference to time is contained in the (primitive)
meaning the object properties, it offers no resource of precisely expressing 
time-related assertions on the level of individuals. 
This is possible with TQC. Each A-Box statement in which a continuant 
is involved is necessarily time indexed. Statements that range about continuants
at several time instants require T-Box expressions, as classes of TQCs have to be built.
More advanced querying, e.g. involving nested time intervals are not supported.
They may require an additional rule language, which, however lies beyond the scope
of this paper. 

The decision whether to prefer TR or TQC as basis of an OWL version of BFO2, will 
have to been taken after more case studies and in dialogue with the BFO user community. 










   

