\subsection*{Overview}
Our goal here is to find an optimal way to represent, distinctively, the three temporal relatedness patterns TGR, PGR, and PSR in OWL ontologies. 
Two competing approaches 
are introduced and discussed, \emph{Temporalized Relations} (TR) and \emph{Temporally Qualified Continuants} (TQC). Whereas TR integrates time 
aspects into the definition of OWL object properties, TQC provides a way to model all instantiations of continuant classes as temporal slices of continuants. 
The approach pursued in this paper can be summarized as follows:

\begin{itemize}
\item
Formulation of examples and competency questions which represent typical representational tasks and reasoning requirements; 
\item
Formalization of the TR approach;
\item 
Formalization of the TQC approach;
\item 
Representation of examples by means of both TR and TQC; 
\item 
Evaluation of TR and TQC inlight of the identified competency questions. 

\end{itemize}


\subsection*{Examples and Competency Questions}
 
The following Examples (EX) and competency questions (CQ) will be used for the comparison of the two representational approaches. 
%They address  phased sortals, changing location at individual level (iii) class-level queries on anatomical objects (iv) class-level queries in which other than mereological relations occur (e.g. inherence, participation)}
For each temporal kind of relatedness(TGR, PGR, PSR), examples are created according to the following criteria:  
class-level (c), individual-level (i) mereological (m+), non-mereological (m-), participation (p), 
with quantification over the participant (p1) and over the process (p2). 
In addition we will provide a class-level and an individual-level examples for a non-rigid classes, i.e. OWL classes whose members may not instantiate this class for all of their lifetime, such as \emph{Student} or \emph{Embryo} \cite{OntoClean}. 
For each CQ, its conformance to the corresponding EX is analyzed by the authors and decided of whether it is satisfiable with regard to the premise stated in the corresponding EX. Satisfiability of the example $i$ means that there is at least one truth-making interpretation of the logical theory, consisting of the ontology $O$, together with the example $EX_i$ and the competency querstion $CQ_i$.  \todo{copy the modified wording of the UCs and CQs to the Excel table}
 
 
\begin{itemize}
\item STR\_cm+: Every tooth is part of an organism at some time. \\
Challenged by: 
\\ ``tooth for which there is a time it is not part of any organism '': 
$\rightarrow$ \emph{satisfiable} \\
%``tooth that is part of something that is not an organism or organism part'': \emph{satisfiable} 


\item STR\_cm-: Every apple is green at some time
\\
Challenged by: 
%\\ ``apple that is never green'': \emph{unsatisfiable} 
\\
``apple that at some time has no other color than read'': 
$\rightarrow$ \emph{satisfiable} 


\item STR\_cp1: Every vertebrate participates in some birth process at some time\\
Challenged by: \\  
``Vertebrate that at some time does not participate in a birth process'': 
$\rightarrow$ \emph{satisfiable} 



\item STR\_cp2: Every fecundation has some spermatozoon as participant at some time
\\
Challenged by: \\ 
``Fecundation event for which there is never any spermatozoon that participates'': 
$\rightarrow$ \emph{unsatisfiable} \\
%``Phase of fecundation in which there is no spermatozoon participating'': \emph{satisfiable} 

\item STR\_im+: ``The blood specimen bs430912 is in the lab996 during the whole time interval 2013-10-20.'' 


Challenged by: \\ ``The blood specimen bs430912 is in the lab lab100 during the whole time interval 2013-10-20. The two labs are in different places'' 
$\rightarrow$ \emph{unsatisfiable}



\item STR\_im-: ``Joe's left ankle is swollen during the whole time interval 2013-10-20''


Challenged by: 
``There is a time within the interval 2013-10-20 in which Joe's ankle is not swollen '' 
$\rightarrow$ \emph{unsatisfiable}



\item STR\_ip:  ``Mary participated in the process of Mary's birth''

Challenged by: \\``There is a time in which Mary is not a participant in any birth process''  
$\rightarrow$ \emph{satisfiable}

%%%%%%%


\item PGR\_cm+: ``Every red blood cell always includes some oxygen molecule''
Challenged by: \\``There are red blood cells without oxygen molecules''  
$\rightarrow$ \emph{unsatisfiable}


\item PGR\_cm-: ``Every electronic health record is always in some computer storage medium''

Challenged by: \\``There are electronic health records that are not in any computer storage medium''  
$\rightarrow$ \emph{unsatisfiable}


\item PGR\_cp1: ``Every organism participates in some ecologic process at all times while alive''

Challenged by: \\``There are organisms that do not participate in any ecologic process at certain times''  
$\rightarrow$ \emph{unsatisfiable}


\item PGR\_cp2: ``Every breathing process (in mammals) has at every time in its temporal extent some volume of air as participant'' 


Challenged by: \\``There are parts of breathing processes (in mammals) in which no air volume participates''  
$\rightarrow$ \emph{unsatisfiable}


\item PGR\_im+: ``Mary's blood always has as part some blood cell ''
Challenged by: \\``There is a time in which Mary's blood is devoid of blood cells''  
$\rightarrow$ \emph{unsatisfiable}

\item PGR\_im-: ``Joe's electronic health record always resides in some electronic storage medium''

Challenged by: \\``There is a time in which Joe's electronic health record is on hard disk A and another time in which it is only on a different storage device''  
$\rightarrow$ \emph{satisfiable}


\item PGR\_ip:  ``Joe's breathing process has at every time during which it is occurring some volume of air as participant''

Challenged by: \\``Joe's breathing process is without air at some time''  
$\rightarrow$ \emph{unsatisfiable}

%%%%%%%%%%%

\item PSR\_cm+: ``Every human has a brain and it is always the same brain ''

Challenged by: \\``There is a time in which Joe does not have his original brain''  
$\rightarrow$ \emph{unsatisfiable}

%\todo[inline, size=\small]{``This is rather tricky with TQC. One could express it as: for all temporal 
%instances of Human there is always some temporal instance of brain related. For all human instance which has the same "max" 
%all related brain instances must also have the same "max" ''}


\item PSR\_cm-: ``Every mammal has the same biological sex during its entire life''

Challenged by: \\``A mammal that is biologically male at some time and biologically female at some other time''  
$\rightarrow$ \emph{unsatisfiable}

 
\item PSR\_cp1: ``Every organism participates in its life, at every time at which it is alive, and it is always the same life''

Challenged by: \\``Joe participates in two different lives''  
$\rightarrow$ \emph{unsatisfiable}


\item PSR\_cp2: ``Every biological life has some organism as participant, and it is the same organism at every time during the course of this life''

Challenged by: \\``A life that has for some time a human as participant and for some other time another organism as participant''  
$\rightarrow$ \emph{unsatisfiable}


\item PSR\_im+: ``Mary has `Mary's brain' as a proper part during her lifetime''

Challenged by: \\``There is a time during which Mary has no brain''  
$\rightarrow$ \emph{unsatisfiable}


\item PSR\_im-: ``Tibbles, a male cat, has the same sex during his life'' 

Challenged by: \\``Tibbles is female at some time during his life''  
$\rightarrow$ \emph{unsatisfiable}


\item PSR\_ip: ``Joe participates in his life and not in any other life''

Challenged by: \\``Joe participates in a human life and part of his life is a cat life''  \todo{BS: not very convincing}
$\rightarrow$ \emph{unsatisfiable}


%%%%%%%%%%%


\item NR\_t: ``Every fetus is an embryo at some time. 
% logically irrelevant according to BS: Nothing can be both an embryo and a fetus at the same time''

Challenged by: \\ ``There are fetuses that were never embryos''  
$\rightarrow$ \emph{unsatisfiable}


\item NR\_i: ``John was a medical student from 2001-10-01 to 2007-06-30. Every person who is a medical student was a high school student at some time''

Challenged by: \\ ``John was never a high school student''
 $\rightarrow$ \emph{unsatisfiable}

%For the implementation of the examples and competency questions as OWL axioms see supplementary data.
 
%
%Development process of gastrulation which starts with one anatomical part and goes on to another anatomical part? See gastrulation wikipedia page. 
%I see this part. Which phases/stages could my organism be in?
%
%During neural development the cells are doing something specific. Existence of process defines phase.
%
%\todo[inline, size=\small]{Alan has a worked out example of the phases of life of some moth organism.}
%\todo[inline, size=\small]{Embryo has this part but the fetus does not have that part.}
%\todo[inline, size=\small]{Joe's income when he was a medical student was lower than his income after he graduated.}
%\todo[inline, size=\small]{An assertion about joe as a medical student shouldn't make it on to Joe in general.}
%

\subsection*{Formalization of both TR and TQR}
Both approaches are expressed in first order logic (FOL). FOL expressions are then 
transformed in a way that essentially preserves the semantics on the basis of the BFO 2 technical specification \cite{BFO2:ref}, but  
use exclusively binary relations. From the resulting formulas, corresponding OWL axioms can then be derived.  

\subsection*{OWL Implementation and evaluation methods}
The TR approach has already been implemented as an OWL ontology \cite{BFO2:Graz}. We develop the TQR ontology by removing all TR-specific relations 
from the TR ontology and adding 
the additional content required for the TQC solution, using Prot\'eg\'e 4.3. These OWL files are then imported into the specific ontologies that 
implement the examples. We then assess the two sets of treatment of these examples in light of the above introduced competency questions formulated as DL queries. 
For each example and either the TR or the TQC ontology we investigate whether (i) the corresponding approach captures the intended semantics, 
(ii) whether it provides a correct answer to the related competency question, (iii) whether the solution is user-friendly, and (iv), whether it has a good computational performance.   
  
\end{itemize}