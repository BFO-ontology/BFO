%% vim: spelllang=en
\documentclass{ao2e}
\usepackage[utf8]{inputenc}
\usepackage{times}
\usepackage[T1]{fontenc}
\usepackage[british]{babel}
\usepackage[usename,dvipsnames]{xcolor}
%\usepackage[]{hyperref}
%\usepackage{varioref}
%\hypersetup{colorlinks=true,urlcolor=black, linkcolor=black,
%citecolor=black,pdftitle=Expressing time-dependent relations through temporal qualifications,pdfauthor=Niels Grewe et al,unicode=true}
%\PrerenderUnicode{ï}
\usepackage[babel]{csquotes}
%\bibliographystyle{plainnat}
\usepackage{pgf,tikz}
\usepackage{todonotes}
\usepackage{bussproofs}
\usepackage{listings}
\usetikzlibrary{positioning,shapes,shadows,arrows,backgrounds}
\usepackage{covington,booktabs}
%\usepackage{multirow}
\usepackage[style=numeric,citestyle=numeric-comp,backref=false,hyperref=false]{biblatex} 
%\usepackage[hyperref=true]{biblatex}
%\usepackage{natbib}
\usepackage{amsmath}
\usepackage{enumerate}
\bibliography{temporalized}
\MakeOuterQuote{"}
\MakeAutoQuote{„}{“}
\usepackage{listings}

\firstpage{0}
\lastpage{0}
\volume{0}
\pubyear{0000}

% Shorthands for
%  Instance level relations:
\newcommand{\mirel}[1]{\ensuremath{\mathrm{\mathbf{#1}}}}
%  Class expressions:
\newcommand{\mclass}[1]{\ensuremath{\mathit{#1}}}
%  arity-indexed relations:
\newcommand{\mrel}[2]{\mirel{#1^#2}}
%  binary relations:
\newcommand{\mrelb}[1]{\mrel{#1}{2}}
%  ternary relations:
\newcommand{\mrelt}[1]{\mrel{#1}{3}}
%  DL interpretation function
\newcommand{\dlint}[1]{\ensuremath{#1^{\mathcal{I}}}}
%  ordered pairs:
\newcommand{\pair}[2]{\ensuremath{\langle #1,#2\rangle}}
% TQCs:
\newcommand{\TQC}[1]{\ensuremath{TQC_{\mclass{#1}}}}
% temporary relatedness
\newcommand{\mreltemp}[1]{\mrel{#1}{{Temp}}}
\newcommand{\mrelpg}[1]{\mrel{#1}{{PG}}}
\newcommand{\mrelps}[1]{\mrel{#1}{{PS}}}
\begin{document}
\lstset{
language=SQL,                             % Code langugage
basicstyle=\ttfamily,
columns=flexible,
deletekeywords = { some },
morekeywords = { BEGIN, END, GROUPS, CLASS, OBJECTPROPERTY, FAIL, REMOVE},
sensitive = true
}
\begin{frontmatter}



\title{Expressing time-dependent relations through temporal qualifications}
\runningtitle{Expressing time-dependent relations through temporal qualifications}
\maketitle
\author[HRO]{\fnms{Niels} \snm{Grewe}%
\thanks{Corresponding author: Niels Grewe, Institute of Philosophy, University
of Rostock, 18051 Rostock, Germany.}},
\author[GV]{\fnms{Janna} \snm{Hastings}},
\author[HRO]{\fnms{Ludger} \snm{Jansen}},
\author[XXX]{\fnms{Fabian} \snm{Neuhaus}},
\author[SUNY]{\fnms{Alan} \snm{Ruttenberg}},
\author[LBL]{\fnms{Chris} \snm{Mungall}},
\author[GRZ,FB]{\fnms{Stefan} \snm{Schulz}}
\address[GRZ]{Institute for Medical Informatics,
Statistics and Documentation, Medical University of Graz, Austria}
\address[FB]{Institute of Medical Biometry and Medical Informatics, University Medical Center Freiburg, Germany}
\address[HRO]{Institute of Philosophy, University of Rostock, Germany}
\address[GV]{Institute of Philosophy, University of Geneva, Switzerland}
\address[LBL]{Genomics Division, Lawrence Berkeley National Laboratory, Berkeley, CA, USA}
\address[SUNY]{State University of New York, Buffalo,
NY, USA}
\address[XXX]{XXX}

\runningauthor{N. Grewe et al.}
\begin{abstract}
We discuss the difficulties of representing different forms of temporal
relatedness in OWL 2, given its limitation to binary relationships. Based on the
fact that temporary relatedness is important for modelling domains that are
characterised by dynamic phenomena, we propose a reification scheme to express
temporary and permanent relatedness that combines semantic accuracy, ease of use
and ontological rigour. Central to this scheme is the notion of temporal
qualification that provides a means to abstract from parts of the history of a
continuant.
\end{abstract}

\begin{keyword}
relation\sep time\sep stage\sep phase\sep top-level ontology\sep BFO\sep
reification\sep ternary relation\sep OWL 2
\end{keyword}
\end{frontmatter}
\section{Introduction}

This paper addresses an ontological puzzle that has arisen in the
Open Biomedical Ontologies (OBO) community and in the current BFO 2 redesign
project.  However, it seems to be of general interest, as it addresses a
weakness of all OWL ontologies that link continuant entities by binary relations
(OWL object properties) . Just as most application ontologies, most OBO
ontologies embrace
three-dimensionalism as their ontological framework, introducing a top level
distinction between continuants and occurrents, or endurants vs. perdurants,
which is present in top-level ontologies such as BFO (\cite{BFO1:ref}), DOLCE
(\cite{DOLCE:ref}) and GFO (\cite{GFO:ref}),
although not all of them use the same terminology to express this
distinction.

But not only do entities need to be differentiated based on their temporal mode
of existence, temporal aspects also have to be taken into account when
specifying the relationships between those entities. One particular problem
arises when the topic domain in question includes not only static, but also
dynamic aspects of reality. This is the case for most of the life sciences,
where continuants not only persist, but also continuously change over time. The
shape and the constitution of biological organisms are in continuous flux. They
continuously gain and lose parts, qualities, and dispositions. Consequently, any
relational expression which makes reference to a particular continuant can have
different truth values at different times and would therefore be ambiguous if
time were not made explicit in the statement. For example, each of the following
pairs of statements may be true together. The ternary instantiation relation
\mirel{inst^3} relates an individual to a type at a time. 

\pagebreak
\vspace{1\baselineskip}
\noindent\emph{A ripening apple:}
\nopagebreak
\begin{equation}
\begin{split}
\exists a, g, r, t_0,t_1&:\, \mirel{inst^3}(\mclass{Apple}, a, t_0) \wedge
\mirel{inst^3}(\mclass{Green},g,t_0) \\
&\quad \wedge
\mirel{inst^3}(\mclass{Apple},a,t_1) \wedge
\mirel{inst^3}(\mclass{Red},r,t_1) \\
&\quad\wedge
\mirel{bearerOf}(a,g,t_0) \wedge \neg\mirel{bearerOf}(a,r,t_0) \\
&\quad\wedge \neg\mirel{bearerOf}(a,g,t_1) \wedge \mirel{bearerOf}(a,r,t_1)
\end{split}
\end{equation}
\noindent\emph{Situation before and after a heart transplant}

\begin{equation}
\begin{split}
\exists a, b, t_0, t_1&:\, \mirel{inst^3}(\mclass{Heart},a,t_0)
\wedge \mirel{inst^3}(\mclass(Body,b,t_0)
\wedge \mirel{inst^3}(\mclass{Heart},a,t_1) \\
&\quad \wedge \mirel{partOf}(a,b,t_0)
\wedge \neg\mirel{partOf}(a,b,t_1)
\end{split}
\end{equation}

In these examples continuants of different kinds are related at different time
instants. But time-indexed relations are also relevant when connecting
continuants with processes: 

\vspace{1\baselineskip}
\noindent\emph{The soprano Mary participating in a performance of Beethoven’s
9th symphony, during the 4th movement only:}

\begin{equation}
\begin{split}
\neg&(\mirel{participatesIn}(\text{Mary},
\text{Beethoven\_9th\_Symphony\_Performance\_1234},
\text{2012-01-01:20:05})) \\
&\wedge\mirel{participatesIn}(\text{Mary},
\text{Beethoven\_9th\_Symphony\_Performance\_1234}, \text{2012-01-01:20:55}) 
\end{split}
\end{equation}
As we see, a time index is necessary for many statements asserting a relation
that changes through time, and this includes many statements involving an
independent continuant, such as in the above examples involving apple\_123, Joe
or Mary etc; though there are some relations where the temporal order is
implicitly clear (for example, in the relation \mirel{derivesFrom}, where it is
clear that the first relatum came into existence before the second relatum).

Top-level ontologies thus need to take into account temporality in order to
provide an accurate model for the interrelations of continuants. Unfortunately,
if one strives to promote the use of top-level ontologies in application
contexts because of their benefits for ontology re-use, data integration or
overall accuracy, complying with this need can become quite complex. This is
because ontological accuracy, alone, is often not the sole guiding principle in
ontology design. Instead, at least two additional issues need to be taken into
account:

\begin{enumerate}
\item The effort required on part of the ontology engineers: In many cases
they are domain experts not trained in formal logic and will have difficulties
if the modelling is not intuitive. This a major reason why four-dimensionalism,
regardless of its philosophical merits, is not the ontological approach of
choice (as long as the topic domain is not fraught with problems of relativity):
Most of us do not normally think of Mary or apple \#123 as
four-dimensional space-time worms, but rather as three-dimensional things. 
\item The representation formalisms available, together with the tools (editors,
reasoners) that support them: These formalisms need to compromise between
expressiveness and computational complexity. Currently, OWL 2 ($\mathcal{SROIQ}$) is the
most expressive formalism that is standardized, and for which
authoring tools and reasoners are widely available. OWL 2 is also the favourite
formalism embraced by the OBO community. 
\end{enumerate}

OWL 2 allows only binary relations between individuals, called
\emph{object properties}. But while there has been work on expressive description
logics that try to transcend this limitation (\cite{Calvanese:1997}) and also on description logics
that explicitly account for temporality (e.g. \cite{Wolter:2001}) , there is no strong push
towards standardisation of those formalisms, and tools suitable for end users
are not readily available. It has thus been acknowledged that there is a need
for solutions that work within the confines of present technologies
(\cite{Welty:2006}).
Time-indexed relationships have been introduced as instance-level relations into
the OBO Relation ontology (\cite{OBO:RO}). In the current work on a new release of the Basic
Formal Ontology, BFO 2, ternary (time-indexed) instance-level relations have
been formulated for the first order logic variant of the ontology, though they
do not carry over into the OWL version yet.

The question we address in this paper is whether the favoured
compatibilist approach that emphasises the validity of the three-dimensionalist
view of reality is reconcilable with the requirements of
representing knowledge about changing relationships between entities. We will
analyse the consequences of this for the creation of a compatible OWL version of
BFO 2. The paper is organized as follows: we specify the assumptions we make
about the three-dimensional framework that is used as a basis to represent
change over time. This is done in first order logic. We then extend that
framework by introducing temporally qualified continuants, and show how binary
relations between temporally qualified continuants can be defined based on the
underlying relationships between continuants as vehicle to express change in
OWL. In a first glance, these entities might look like four-dimensional entities
in sheep's-clothing. However, we will show that these temporally qualified
continuants are just introduced as \emph{façon de parler}, and thus are
ontologically innocuous or can at least be contained in a special fragment of the ontology.
\section{Basic notions}
Throughout the paper, we will use „\mirel{rel}“ as an example and general placeholder
for any relation that developers might want to use in an ontology. If this
relation is originally a ternary relation $\mrelt{rel}(a, b, t)$ with explicit time
index, the ontology engineer faces the problem how to express it due to the
limitations of OWL, using a binary object property \mrelb{rel}. Ignoring this problem,
ambiguous interpretations arise, as exemplified in the following class
expression (formulated in Manchester Syntax, \cite{Man:Syntax}):
\begin{equation}
\mclass{A}\; \mathtt{subClassOf}\; \mrelb{rel}\; \mathtt{some}\; \mclass{B}
\end{equation}

Different interpretations of this expression are possible, based on the temporal
strength of the relational term „\mirel{rel}“, for which we will now suggest a
rendering in FOL. For this purpose, we use the time-indexed instantiation
relation \mrelt{inst} that holds between an individual and a class at some time
$t$ if
and only if the individual instantiates the class at $t$. In addition, we use the
relation of temporal inclusion \mirel{within}. Temporal inclusion holds between
temporal instants if and only if they are the same instant; temporal inclusion
for temporal regions is as expected.


We here define three temporal strengths, viz. temporary relatedness, permanent
generic relatedness and permanent specific relatedness:

\begin{equation}
\begin{split}
\mclass{TemporarilyRelated}(\mclass{A},\mclass{B}) =_{def}&\;
\forall a, t:\; \mrelt{inst}(\mclass{A}, a, t) \\
&\ \rightarrow
\exists b, t^\prime:\;(\mrelt{inst}(\mclass{B},b,t^\prime) \wedge
\mrelt{rel}(a,b,t^\prime) \wedge \mirel{within}(t^\prime,t))
\end{split}
\label{eq:temporarily}
\end{equation}


Informally: for all a instances of \mclass{A} there is some time $t$ and some instance $b$ of
\mclass{B} such that $a$ is related to $b$ at $t$. Examples: 
\begin{enumerate}[(a)]
\item for all apple seeds there is
some apple such that the seed is part of the apple at some time. 
\item for all
trees there is some leaf such that the leaf is part of the tree at some time.
\end{enumerate}

\begin{equation}
\begin{split}
\mclass{PermanentlyGenericallyRelated}(\mclass{A},\mclass{B}) =_{def}&\;
\forall a, t:\; \mrelt{inst}(\mclass{A}, a, t) \\
&\ \rightarrow
\exists b:\;(\mrelt{inst}(\mclass{B},b,t) \wedge
\mrelt{rel}(a,b,t))
\end{split}
\label{eq:generically}
\end{equation}

Informally: for all instances $a$ of \mclass{A} there is, at all times $t$ that
$a$ exists,
some instance $b$ of \mclass{B} such that a is related to $b$ at $t$, but not necessarily
always the same $b$ at all times $t$. Examples:
\begin{enumerate}[(a)]
\item all cells have a water molecule as
part at all times, but not always the same water molecule.
\item every bacteria colony has some bacteria as parts at all times, but not
always the same bacteria.
\end{enumerate}


\begin{equation}
\begin{split}
\mclass{Permanently}&\mclass{SpecificallyRelated}(\mclass{A},\mclass{B}) =_{def}\;
\forall a, t:\; \mrelt{inst}(\mclass{A}, a, t) \\
&\ \rightarrow 
\exists b:\;\big(\mrelt{inst}(\mclass{B},b,t) \wedge 
\mrelt{rel}(a,b,t))
\\
&\quad\quad \wedge \forall t^\prime: (\mrelt{inst}(\mclass{A},a,t^\prime)
\rightarrow (\mrelt{rel}(a,b,t^\prime) \wedge
\mrelt{inst}(\mclass{B},b,t^\prime))\big)
\end{split}
\label{eq:specifically}
\end{equation}

Informally, for all instances $a$ of \mclass{A} there is, at all times $t$ that $a$ exists, an
instance $b$ of \mclass{B} such that a is related to $b$ at $t$; in this case it is always the
same $b$ at all times $t$. Examples:
\begin{enumerate}[(a)]
\item a human being has a brain as part at all times, and it is necessarily the same brain.
\item a radioactively marked molecule of DNA has the radioactive isotope as part
at all times, and it is necessarily the same atom.
\end{enumerate} 


Our working hypothesis will be that interpretations, „\emph{temporary
relatedness}“ (\ref{eq:temporarily}) and „\emph{permanent relatedness}“
(\ref{eq:generically}--\ref{eq:specifically}) are relevant in the context of
application ontologies. %It is still subject to debate to what extent the
%distinction between specific and generic relatedness matters.
 In what follows,
we will attempt to sketch solutions for expressing these interpretations in OWL
2. Our primary point of reference will be the BFO top-level ontology, the
second, largely extended release being under current development, with a
preview release available at \cite{BFO2:Graz}.  Nevertheless
the proposal sketched here could potentially be implemented in other top-level
ontologies that are sufficiently similar to BFO.

\section{Related work}
\subsection{Connection to the SNAP/SPAN distinction in BFO 1}


BFO has traditionally maintained a strong distinction between 3D and 4D accounts
of reality (\cite{BFO1:ref}). According to BFO, a 3D description of reality exposes certain
truths about spatial and spatio-temporal phenomena, e.g. between ordinary
objects that persist through time and the processes in which these objects are
involved, while a four-dimensionalist view coalesces both into a single
spatiotemporal account where, for example, processes involving objects are
understood as something like „space-time worms“.  But despite making this
distinction, BFO maintains that these are not two incompatible descriptions of
reality but instead that the continuant and the occurrent views represent
complementary perspectives. Consequently, the ontological account of BFO is
partitioned into two kinds of constituent ontologies. On the one hand, there are
a series of 3D ontologies, which can basically be thought of as „snapshots“ of
reality at a given point in time and are hence called SNAP ontologies. On the
other hand, the overarching four-dimensional picture is provided by a so-called
SPAN ontology, to which all entities from SNAP ontologies are related by way of
trans-ontological relations.


This highlights a significant mismatch between the theoretical framework of BFO
and the constraints of OWL 2: the BFO solution is to use time-indexed SNAP
ontologies for which a certain relational assertion holds. A similar mechanism
is not specified for OWL 2 ontologies (though one can distinguish versions of an
ontology) nor can the BFO approach be adopted as an informal convention if the
ontologies are supposed to support automated reasoning. This is because of the
following issue: assume that there are two SNAP ontologies, $O_0$ and $O_1$, each
describing whether the relation \mirel{rel} holds between two objects $a$ and $b$ at
different points in time. Now, if in $O_0$ „$\mirel{rel}(a, b)$“ is true, and in
$O_1$, „$\mirel{rel}(a, b)$“ is false, we derive a contradiction once we construct a SPAN ontology
$O_s$ that references both $O_0$ and $O_1$, due to the fact that the object property
corresponding to \mirel{rel} shares the same namespace in all three ontologies.
Consequently, if one chooses to produce a single ontology in which both SNAP and
SPAN ontologies co-exist, there is room for only a single snapshot in that
ontology.


This problem could be mitigated by introducing explicitly namespaced object
properties, but this is highly impractical because it results in an extreme
proliferation of relations, one for each point in time. Additionally, it is not
ontologically sound to interpret these object properties as relationship
universals, because they make claims about universals which are only applicable
at one single point in time, which would be a very strange thing to claim about
an universal unless it holds only derivatively.
We thus need to look for alternative solutions to this problem that still
capture the intended meaning of the SNAP/SPAN distinction, but are manageable
for ontology engineers using OWL 2 or other languages from the description logic
family.
\subsection{Histories in BFO 2}

In BFO 2, work is under way to sketch out a more detailed theory of the
relationship between SNAP objects and SPAN processes they participate in.
Specifically, BFO 2 make the assumption that for each material object there
exists a special process, the \emph{history} of the object, which encompasses 
„the totality of processes taking place in the spatiotemporal region occupied by
the entity.“ (\cite{BFO2:ref}) This means that there is a one to one correspondence
between objects (on the SNAP side) and certain processes, which effectively
provides a „bridge“ between the 3D and the 4D perspective. No complete formal theory
of histories, which have previously been described in \cite{cornucopia}, is available as of yet.
\subsection{Conventional Modellers’ Strategy for Temporalised Relations in OWL 2}

As explicit semantics for modelling temporal dynamics are not available in OWL
2, modellers tend to implicitly treat object properties as committing to a 
„for all times“ interpretation in order to avoid obvious problems with the
entailed models. For instance, in an anatomy ontology like the FMA, the object
property \mirel{hasPart} is transitive, and used in axioms such as
„$\mclass{Lung}\;\mathtt{subClassOf}\;\mirel{hasPart}\;\mathtt{some}\;\mclass{LobeOfLung}$“
and „$\mclass{LobeOfLung}\;\mathtt{subClassOf}\;\mirel{hasPart}\;\mathtt{some}\;\mclass{BronchiopulmonarySegment}$“

If the underlying interpretation were „for some time“, transitivity of the
binary \mirel{hasPart} could no longer be taken for granted, as two
\mirel{hasPart} assertions to be chained could belong to two different SNAP
ontologies.

There is thus good reason to subscribe to interpretation (\ref{eq:generically}),
as this is the only available interpretation that has a possibility of being
consistent with the semantics. But ontology builders are usually not aware of
the fact that they have implicitly substituted the interpretation function
(which maps syntactic constructs of OWL 2 to intended models) with a
temporalised variant, regardless of whether they would have intended the
permanent or temporary parthood variant if they had been aware thereof.

Obviously, the interpretation must be equivalent to the OWL 2 direct semantics
(\cite{OWL2:direct}) in as far as it preserves syntactical structure and inferences and does not
lead to additional expressivity. But it has – to our knowledge – never been made
explicit what this substitution might consist of.  This is even more significant
as this does not constitute at all a side issue or an idiosyncrasy of biomedical
ontologies. On the contrary, virtually all OWL ontologies contain axioms on
classes of continuants using binary object properties and leave the exact
interpretation unexplained. 

To address this mismatch and try to understand it better we sketch here a
possible elucidation  which consists in a modification of the interpretation
function. The general strategy of this interpretation is to augment the
interpretations of class members and object properties in the OWL model with an
additional time index $t$ which specifies that the entity in question exists
(object property holds) at $t$. Class instances then become pairs and object
property instances triples. In order to keep the surface grammar and overall
semantics intact, the interpretations of all OWL axioms will be prepended with a
conditionalized universal quantification over $t$  hat specifies that the axiom
should hold at all times that the entity in question exists.  Time instants are
hereby external to the domain. For example, the interpretation of a class
assertion axiom that asserts that $a$ is an instance of class \mclass{C}, as long
as $a$ exists, would then read (domain~$\Delta$, interpretation~$\cdot^\mathcal{I}$):

\begin{equation}
\forall t:\;\pair{\dlint{a}}{t}\in \dlint{\Delta} \rightarrow
\pair{\dlint{a}}{t} \in \dlint{\mclass{C}}
\end{equation}

We implicitly assume that $\Delta$ contains individual/time-point pairs only for
those times at which an individual exists. Notably, this is only sufficient to express rigid instantiation: Whenever an individual exists at
all, it is also a member of the class it instantiates. The interpretation of
temporality-sensitive relations will become clear when we spell out the semantic
rules of existential quantification and value restriction, both of which assert
permanent generic relatedness because they apply existential quantification over the
object property range so that at each point in time a different individual of
class \mclass{B} can serve as a relatum. We will use the canonical structural syntax
(\cite{OWL2:structural})
to ease comparison with the specified semantics (\cite{OWL2:direct}).

\begin{description}
\item[Existential quantification ($\mrelb{rel}\;\mathtt{some}\;\mclass{B}$)]
\begin{equation}
\begin{split}
\dlint{\text{ObjectSomeValuesFrom}&(\mrelb{rel},\mclass{B})} =_{def}\\ &\quad
\{\pair{\dlint{a}}{t}\in \dlint{\Delta}\,|\; \exists b:\;\langle\dlint{a},b,t\rangle
\in \dlint{\mrelt{rel}} \wedge \pair{b}{t} \in \dlint{\mclass{B}}\}
\end{split}
\end{equation} 
\item[Value restriction ($\mrelb{rel}\;\mathtt{only}\;\mclass{B}$)]
\begin{equation}
\begin{split}
\dlint{\text{ObjectAllValuesFrom}&(\mrelb{rel},\mclass{B})} =_{def}\\ &\quad
\{\pair{\dlint{a}}{t}\in \dlint{\Delta}\,|\; \forall b:\;\langle\dlint{a},b,t\rangle
\in \dlint{\mrelt{rel}} \rightarrow \pair{b}{t} \in \dlint{\mclass{B}}\}
\end{split}
\end{equation}
\end{description}

In OWL object property assertions the time index is bound through universal
quantification again:

\begin{equation}
\text{ObjectPropertyAssertion}(\mrelb{rel},a,b) =_{def}\;\forall
t:\;\pair{\dlint{a}}{t} \in \dlint{\Delta} \rightarrow \langle
\dlint{a},\dlint{b},t\rangle \in \dlint{\mrelt{rel}}
\end{equation}

Hence, object property assertions specify permanent relatedness. Disregarding
the difference between specific and generic permanent relatedness for the time
being, this interpretation of 
OWL 2 at least successfully mimics the semantics of class level
relations intended by the relations ontology (RO, \cite{OBO:RO}) and allows us to think of
the syntactical forms represented in table \ref{tab:syntaces} as equivalent.
\begin{table}
\caption{Syntactical representations of (permanent) relatedness expressions}
\label{tab:syntaces}
{
\begin{tabular}{p{10.9em}cp{10.5em}}
\toprule
\parskip=0cm
\parbox{10.9em}{\centering OBO Syntax} & OWL (Manchester Syntax) & \parbox{10.5em}{\centering First Order Logic} \\
\midrule
\texttt{$[$Term$]$}\par
\texttt{id:} \mclass{A}\par
\texttt{relationship:} \mrelb{rel} \mclass{B} &

\parbox[t][1.5em][c]{11.2em}{\centering $\mclass{A}\;\mathtt{subClassOf}\;\mrelb{rel}\mathtt{some}\mclass{B}$} &

$\forall a,t:\;\mrelt{inst}(A,a,t)\rightarrow$\par
$\quad \big(\exists b:\;\mrelt{inst}(B,b,t)\;\wedge$\par
$\qquad\;\mrelt{rel}(a,b,t)\big)$\\
\bottomrule
\end{tabular}
}
\end{table}

This approach  also retains standard transitivity semantics of OWL 2 object
properties, so that quantification over time maintains transitivity of the
relation in question.
This can be shown, e.g. for the transitive relation \mirel{hasPart}: if an organism has
some heart at any time, and if this heart has some heart valve at any time, then
the organism has some heart valve at any time:

\begin{prooftree}
\AxiomC{$\mclass{A}\;\mathtt{subClassOf}\;\mrelb{hasPart}\;\mathtt{some}\;\mclass{B}$}
\AxiomC{$\mclass{B}\;\mathtt{subClassOf}\;\mrelb{hasPart}\;\mathtt{some}\;\mclass{C}$}
\BinaryInfC{$\mclass{A}\;\mathtt{subClassOf}\;\mrelb{hasPart}\;\mathtt{some}\;\mclass{C}$}
\end{prooftree}
And while there is nothing to be gained by actually modifying OWL 2 to use this
interpretation, it is very important that ontology engineers are aware of the
implications of their modelling decisions with regard to relations that are
sensitive to the issue of temporal strength. However, this approach
still has the consequence that „\emph{temporary relatedness}“ cannot be
expressed directly in an OWL 2 ontology, so we need to look for more involved
solutions to the problem.

\subsection{Proposed 4D-ist solutions}
\subsubsection{Reification}

A common strategy to work around the limitations of description logics is to
represent ternary relations through reification. Reification involves the
introduction of a class $\mclass{C_\mrelt{rel}}$ for each ternary relation
\mrelt{rel}. The relata of \mrelt{rel}
are then connected to instances of $\mclass{C_\mrelt{rel}}$ by three new binary
relations \mrelb{R_1}, \mrelb{R_2},
\mrelb{R_3}. The instance-level assertion
$$
\mrelt{rel}(a,b,t)
$$

would then be transformed into the following statement:\todo{Changed from list of statements to proper axiom}
\begin{equation}
\exists x: \mclass{C_\mrelt{rel}}(x) \wedge
\mrelb{R_1}(x,a) \wedge
\mrelb{R_2}(x,b) \wedge
\mrelb{R_3}(x,t) 
\end{equation}

Such proposals have, with a varying degree of sophistication, seen quite a bit
of dissemination in the ontology engineering community (\cite{ODP:nary}), but they suffer
from unavoidable drawbacks. Most obviously, they are rather complex. This bears
the risk of errors in the ontology engineering process and decreases reasoning
efficiency (\cite{Grewe:2010}). To address the complexity problem, it has been suggested to
select reification classes based on what seems ontologically „fitting“ for the
domain of an ontology (\cite{Fiadeiro:2010}).

In fact, what seems to be artificial reification from one perspective could even
be perceived as ontologically sound representation from a different viewpoint.
We might, for example, compare the ternary relational statement
\begin{equation}
\exists x,y,z:\;\mrelb{inst}(\mclass{Aconitase},x) \wedge
\mrelb{inst}(\mclass{Citrate},y) \wedge \mrelb{inst}(\mclass{Isocitrate},z)
\wedge  \mrelt{transforms}(x,y,z)
\end{equation}

with the expression
\begin{equation}
\begin{split}\mclass{Trans}&\mclass{formationProcess}\;\mathtt{and}\\ 
            &\mrelb{hasParticipant}\;\mathtt{some}\;\mclass{Aconitase}
\;\mathtt{and} \\
            &\mrelb{hasInput}\;\mathtt{some}\;\mclass{Citrate}\;\mathtt{and}\\\ 
            &\mrelb{hasOutcome}\;\mathtt{some}\;\mclass{Isocitrate}
\end{split}
\end{equation}

and judge upon superficial inspection that the later is a reification of the
former (where „\mrelt{transforms}“ is represented by the class „\mclass{Trans\-format\-ion\-Process}“). But if one commits to the position that it is useful and
ontologically sound to accept the category of processes into an ontology, the
„reified“ translation might actually be the proper representation of the
ontological fact, i.e. that there is a transformation process going on whenever
aconitase transforms citrate to isocitrate. The relational expression is true
only in virtue of the existence of the process.

It is thus only natural that four-dimensionalism aligns nicely with reification
strategies that try to mitigate the existence of an extra time index for a
relation. Even though we do no share the underlying ontological commitment 
\emph{solely} to four-dimensional entities (for reasons outlined earlier in
this paper and
otherwise, \cite{BFO1:ref}), we still acknowledge that four-dimensionalist
approaches have enormous merits when it comes to tackling this problem.

\subsubsection{Welty/Fikes: Fluents}
The prototypical approach for dealing with temporally changing information in
OWL within a four-dimensionalist framework was provided by 
\cite{Welty:2006}. And while they agree that the 4D  approach is „clearly 
not something that immediately appeals to common sense“, they also claim
that it „gives us another tool to use when solving a practical problem.“ To this
end, they present an ontology that models fluents, i.e. „relations that hold
within certain time interval but not in others.“ This works by considering all
entities as four dimensional entities that have temporal parts (time slices),
such that the material object property assertions hold (synchronously) between
time slices. For example, temporary relatedness could be expressed as in
(\ref{eq:fifteen}).

\begin{equation}
\mclass{Leaf}\;\mathtt{subClassOf}\;\mathtt{inverseOf}(\mrelb{timeSliceOf})\;\mathtt{some}\;
            \mrelb{hasPart}\;\mathtt{some}\;\mrelb{timeSliceOf}\;\mathtt{some}\;\mclass{Tree}
\label{eq:fifteen}
\end{equation}

This is by far one of the most straightforward translations of the
four-dimensionalist commitment, but it suffers from a certain verbosity. This
increases even more if permanent relatedness is concerned. In this case, the
above expression would have to be amended to include a
„$\mathtt{inverseOf}(\mrelb{timeSliceOf})\;\mathtt{only}$“ clause to ensure that
all time slices of the entity are appropriately related to a time slice of the
other entity.

\subsubsection{Zamborlini/Guizzardi: Moments, Relators and Qua-Individuals}
The commitment that some relational expressions are in fact better accounted for
as proper entities is also prominent in Zamborlini and Guizzardi treatment of 
contingent properties \cite{Zamborlini:Guizzardi}. For them, certain „material
relations“ only hold by virtue of a separate truthmaker, the so called
\emph{relator}, which is formed by combining the „qua individuals“ that
partake in the relation. Qua individuals abstract away certain aspects of an
individual so that only that information remains which is relevant for the
individuals participation in the relation. 

Both kinds of entities are examples of „moments“ in their
nomenclature, which are said to inhere in individual entities and can thus be
compared to dependent continuants or occurrents (respectively) in BFO
parlance. But while relators might often be appropriately represented by BFO
processes, the admissibility of qua individuals into BFO might be questionable
since they can hardly be aligned with BFO's realist commitment (where sheer
abstractions could only be regarded as artifacts of a persons though process). 

Underneath the level of qua individuals (e.g. „$\mclass{LeafQuaPartOfTree}$“)
and relators, there is the assumption of an ontology of time slices not unlike
the one in \cite{Welty:2006}, such that
temporal overlap between the qua individuals related by the relator can be
enforced.

Zamborlini and Guizzardi cite as an advantage for this approach that it is capable
of representing the persistence of a relationship across multiple time slices
without mentioning each explicitly (because the relator is associated with the
qua-individual and not its time slice). This is part of a set of requirements
suggested for modelling temporally changing information:

\begin{enumerate}
\item Avoid duplication of the other time slices if one entity partaking in the
relation changes.
\item Provide a consistent ontological interpretation of contingent (non-rigid)
instantiation.
\item Avoid repeating persisting properties for each time slice
\item Ensure that immutable properties of an entity cannot be overridden by a
time slice.
\end{enumerate}

We believe these points to be a good starting point for the evaluation of
any proposal to address the problem of time-dependent relation and should be
used to supplement our initial requirements.

\subsubsection{Gangemi: Descriptions and Situations}
Aldo Gangemi's DnS pattern (\cite{Gangemi:DnS}) deserves mention because it treats
time-dependence of relations as a special case of perspectivity which can be
accounted for by the very heavy-duty reification mechanism of descriptions and
situations. In this case, the suggestion is to use the situation pattern in
order to associate the relata and their temporal context with a common
situation, which is effectively a reified assertion (a proposition). Again, such
entities are figments of the mind and can only be admitted into a realist
ontology such as BFO as such -- rather than being a general way to refer to
arbitrary facts. 

Notably, though, Gangemi reminds us of the fact that OWL 2's $\mathtt{hasKey}$
axiom can be used circumvent the problem of possible duplication of instances
for the same relational $n$-tuple: If a situation $\mclass{S}$ were to use the properties
$\mrelb{hasTimeStamp}$, $\mrelb{hasSubject}$, and $\mrelb{hasObject}$, the axiom
\begin{equation}
\mclass{S}\;\mathtt{hasKey}(\mrelb{hasTimeStamp}, \mrelb{hasSubject},
\mrelb{hasObject})
\end{equation}
would ensure that duplicate entities would be coalesced in the model.

\subsubsection{GFO: Presentials}

The GFO top-level ontology provides yet another way to account for
time-dependent relatedness, which can at least be called „4D inspired“:
Instead of continuants that are present as a whole at every point in time
of their existence, in GFO there are „presentials“ which are present as a whole at
exactly one point in time, thus being analogous to instantaneous time slices
. The diachronic
identity that is a key characteristic of a continuant is then obtained by
postulating that for every individual continuant (in non-GFO parlance) a certain
universal (a „persistant“) exists that is instantiated only by a temporally
contiguous set of presentials, one for every point in time (\cite{GFO:ref}). In our eyes,
this approach is not very attractive for two reasons: it is at odds with the
strong intuition that individual continuants such as human beings exist, and,
second, it requires multiple levels of universals to account for conventional
class level assertions, something that might only be acceptable to a very
limited degree, if at all.\footnote{\cite{Armstrong:USR}, for example, believes that
higher order universals are only justified if they pertain to formal
characteristics of an universal.}
Regarding relations of different temporal strength, GFO has adopted an approach
where relations are reified as „relators“ which serve as contexts that aggregate
the relata as „players“ of certain „relational roles“ (\cite{Loebe:2007}). Additionally, GFO
accounts for different temporal modes of relatedness precisely by distinguishing
between presentials and persistants. Unfortunately, GFO only provides a first
order theory of this framework.

\subsubsection{Bittner/Donnelly: Stages}
\cite{Bittner:Donnelly} suggest an ontologically fitting form of reification of
temporally indexed parthood relations that allows distinguishing between
time-dependent and time-independent parthood relations without referring
explicitly to time. For this purpose, they introduce the
concept of „stages“, understood as instantaneous parts of an occurrent that are
collocated with a continuant that endures over a multitude of such stages.
Therefore a relation
between a stage and whatsoever other entity is never ambiguous with regard to
time, because it only exists at the moment the stage exists. Stages allow for
the distinction between temporary and permanent relatedness, though the scope of
the proposal was restricted to the analysis of mereological relations between
entities or their stages respectively:
\begin{equation}
\begin{split}
\mrelb{temporaryPartOf}&(x,y) \equiv \\ &\exists x_s,y_s:\; (\mrelb{stageOf}(x_s,x)
  \wedge \mrelb{stageOf}(y_s,y) \wedge
 \mrelb{partOf}(x_s, y_s))
\end{split}
\end{equation}
\begin{equation}
\begin{split}
\mrelb{permanentPartOf}&(x,y) \equiv\; (\exists x_s\; \mrelb{stageOf}(x_s, x)
  \wedge \\
 &\forall x_{s'}\exists y_s:\;(\mrelb{stageOf}(x_{s'},x) \wedge
\mrelb{stageOf}(y_s,y)   \wedge
 \mrelb{partOf}(x_{s'},y_s)))
\end{split}
\end{equation}


However, there is no reason not to extend this approach to all ternary relations
that involve continuants and are sensitive to ambiguities because of different
temporal strengths: stages allow for the distinction between temporary and
permanent relatedness for a wide variety of relations, but this explicitly
excludes binary relations between continuants or their stages that carry
implicit reference to time and do not need this kind of treatment. For example 
\begin{equation}
\mclass{AppleTree}\;\mathtt{subClassOf}\;\mrelb{derivesFrom}\;\mathtt{some}\;
\mclass{AppleSeed}
\end{equation}
is not in need of a temporalised re-interpretation because \mrelb{derivesFrom} should,
in its definition, already specify the temporal order of the relata. The derived
entity, that is, needs to temporally succeed the entity from which it is
derived. Another example is the relation between dependent continuants and
independent ones as they always span across the whole existence of the further:
the redness of a given apple is the redness of this apple at all times (as long
as there is redness at all). 

The problem with this approach is, as with the other, the interpretation of a
stage: If they are to be interpreted as „[spatial] regions of minimal temporal
extent“, (i.e., three-dimensional entities) how can they at the same time be
„instantaneous parts of perdurants“?
This seems to confound whether we are talking about four- or three-dimensional
entities. In the end, since material relations between continuants, such as
\mirel{partOf}, are reduced to relations between stages (hence occurrents), this
seems to be a reductive proposal that does not fit our initial requirements.

\section{Temporally Qualified Continuants}
\subsection{Process Profiles and Temporal Qualification}
Put this way, any proposal that wants to take seriously the commitment to
three-dimensional entities and still account accurately and succinctly for
temporally changing information about these entities in a restricted language
such as OWL 2, is at a disadvantage when compared to 4D-ist proposals which fit
such a language more naturally. 

But not all is lost: Incidentally, BFO 2 contains the notion of a
\mclass{ProcessProfile} as a solution to a completely
unrelated problem that can serve as a blueprint for approaching temporalised
relations. In BFO 2, process profiles are used to express what other ontologies
would call properties or qualities of processes. The underlying issue is that
BFO 2 regards processes as dependent entities that do not have qualities
over and above those of the entities they depend on (for example, there is nothing about
the process of running that makes a certain running process „fast“, it is rather
a quality of the runner that induces the fastness). 

Still, BFO 2 can support such informal „process qualities“ talk by translating
it into talk about process profiles. Such process profiles allow us to „focus on some one
structural dimension and ignore, or strip away in a process of selective
abstraction, all other dimensions within the whole process“ (\cite{BFO2:ref}), thus only 
leaving, for example, the components of the running process that indicate its speed
characteristics. 

In a similar vein, we can can apply this idea to the temporally changing
properties of a continuant, and define the \emph{temporal qualification} of a
continuant as the result of regarding the continuant in as far as it exists only
within a certain portion of time. Thus, where a process profile is characterised
by its temporal co-extension with its process, a temporal qualification is
characterised by its spatial co-extension with its continuant over the time
period that it qualifies. They differ, however in that a process profile is a
part of the process, while a temporal qualification cannot be a part of its
continuant.

A temporally qualified continuant (TQC) is thus a way of referring to a continuant
during a portion of time. Formally, we can describe it as a tuple \pair{a}{t}
where $a$ is an instance
of a continuant and $t$ is a portion of time. 

There are several axioms needed to link TQCs to the continuants that they are
temporal qualifications of and to ensure that TQC portions of time do not exceed the
allowed portion of time that the corresponding continuant instance spans over
(most of these are similar to the proposals presented above).
Instantiation of a TQC is thus not time-indexed, while normal instantiation of
continuants is time indexed as in the examples above. 


We will use the notation \TQC{A} to denote the class of temporally qualified
continuants which range over continuants of type \mclass{A}.

\begin{equation}
\forall x:\; (\mrelb{inst}(\TQC{A},x) \rightarrow \exists a,t_0,t_1:\;(
\mrelt{inst}(\mclass{A},a,t_0) \wedge \mirel{equals}(x,a,t_1) \wedge
\mirel{within}(t_1,t_0)))
\end{equation}
The intended meaning of the predicate \mirel{equals} is identity.  
We further introduce a relation \mirel{continuantOf} to link a TQC to the continuant
that it is a TQC of. That is, 
\begin{equation}
\forall x\; \mrelb{inst}(\TQC{A},x) \rightarrow \exists a,t:\;(
\mrelt{inst}(\mclass{A}, a,t) \wedge \mrelb{continuantOf}(a,x))
\end{equation}

In the next sections, we will discuss the representation of the three different
temporal strengths, linking from the standard representation in BFO FOL through
the introduction of temporally qualified continuants to the standard
representation in OWL, showing how these different temporal strengths can be
implemented through this method in a binary relationship framework such as OWL,
though this representation will require relations that refer to ternary
predicates in the FOL model, such as \mrelb{continuantOf}, to remain primitive.
\subsection{Temporary Relatedness}

We rephrase the definition (\ref{eq:temporarily}) given above by inserting
temporally qualified continuants and derive the form that the relationship takes
for temporally qualified continuants and a binary relationship. This is a fairly
transparent translation:
\begin{equation}
\begin{split}
\mclass{TemporarilyRelated}(\mclass{A},\mclass{B})& =_{def}\;
\forall a, t:\; \mrelb{inst}(\mclass{A}, \pair{a}{t}) \\
&\ \rightarrow
\exists b, t_1:\;(\mrelb{inst}(\mclass{B}\pair{b}{t_1}) \wedge
\mrelb{rel}(\pair{a}{t_1},\pair{b}{t_1}) \wedge \mirel{within}(t_1,t))
\end{split}
\end{equation}

We then use the \mrelb{continuantOf} relation and the TQC notation to eliminate
the tuples:
\begin{equation}
\begin{split}
\mclass{Temporarily}&\mclass{Related}(\mclass{A},\mclass{B}) =_{def}\;
\forall x:\; \mrelb{inst}(\TQC{A}, x)
 \rightarrow
\exists a,y,z,t_1:\;(\mrelb{inst}(\TQC{A},y) \wedge \\ & \mrelb{inst}(\TQC{B},z) 
 \wedge \mrelb{continuantOf}(a,x) \wedge \mrelb{continuantOf}(a,y) \wedge
\mrelb{rel}(y,z) 
\end{split}
\label{eq:usesSame}
\end{equation}

This means that the logical form of the expression of temporary relatedness is that
at least one temporal qualification of \mclass{A} is related to some temporally
qualified \mclass{B} instance.
However, we need another axiom to constrain \mrelb{rel} in the above to ensure that the
portions of time are appropriately overlapping, since \mrelt{rel} holds at one time
only:
\begin{equation}
\begin{split}
\forall x,y:\;& \mrelb{rel}(x,y) \rightarrow \exists a,b,t,t_1:\;
(\mirel{equals}(x,\pair{a}{t})\wedge \mirel{equals}(y,\pair{b}{t_1})\\ 
& \wedge \mrelb{continuantOf}(a,x) \wedge \mrelb{continuantOf}(b,y) \wedge
\mrelt{rel}(a,b,t_1) \wedge \mirel{within}(t_1,t))
\end{split}
\end{equation}

Now we have derived a binary expression \mrelb{rel} we are free to use this in OWL
axioms. We introduce the relation \mrelb{hasSameContinuant} between temporally qualified
continuants to express that they are TQCs of the same continuant,
expressed in the above axiom (\ref{eq:usesSame}) as $(\mrelb{continuantOf}(a,x)
\wedge \mrelb{continuantOf}(a,y))$, which allows us to express temporary relatedness in OWL as follows:

\begin{equation}
\TQC{A}\;\mathtt{subClassOf}\;\mrelb{hasSameContinuant}\;\mathtt{some
(}\mrelb{rel}\;\mathtt{some}\;\TQC{B}) 
\label{eq:tqc:temp}
\end{equation}

Additionally, we ensure that sharing a temporal qualification amounts to being
the same continuant:\footnote{Implementers should note that OWL 2 only mandates
this for \emph{named} individuals.}

\begin{equation}
\mclass{A}\;\mathtt{hasKey}(\mrelb{continuantOf})
\end{equation}

\subsection{Usability and simplification}
Since the above axiom (\ref{eq:tqc:temp}) makes a claim about the class \TQC{A},
it is less ideal from a usability perspective. We would rather like to say
something about the target continuant classes \mclass{A} and \mclass{B} in our OWL version, for
ease of use by the end user. Thus, we introduce a new relationship,
\mreltemp{rel}, which obtains between continuants, and should be interpreted as follows:


\begin{equation}
\begin{split}
\mclass{A}\;&\mathtt{subClassOf}\;\mreltemp{rel}\;\mathtt{some}\mclass{B}\;\rightarrow\\
&\TQC{A}\;\mathtt{subClassOf}\;\mrelb{hasSameContinuant}\;\mathtt{some}\;(\mrelb{rel}\;\mathtt{some}\;
\TQC{B}) 
\end{split}
\end{equation}

Unfortunately, the above statement cannot be formulated in OWL 2 due to its
strict constraints on object properties. Neither can it be implemented in a
rule language, since it would induce the generation of new individuals in its
consequent, which violated DL safety. Still there is an avenue for hiding the
complexity by using a macro processing engine, such as OPPL (\cite{OPPL}), in
which processing instructions such as these could be employed:

\begin{lstlisting}
?x:CLASS[subClassOf Continuant],
?y:OBJECTPROPERTY?MATCH("temporarily_(.*)"),
?z:CLASS[subClassOf Continuant]
SELECT ?x subClassOf ?y some ?z
BEGIN
  ADD ?x subClassOf continuantOf some ?y.GROUPS(1) 
    some hasContinuant ?z,
  REMOVE ?x subClassOf ?y some ?z
END;
\end{lstlisting}
This can easily be adopted or parameterised for other axiom types or types of temporal
sensitivity (e.g. permanent generic relatedness) Additional measures that alleviate the burden of this approach would be making
\mrelb{rel} a sub-object-property of \mreltemp{rel}, which quite natural and
obvious: If something is related at all times to some entity, it is related to
that entity at some time.
\subsection{Permanent Generic Relatedness}
Permanent generic relatedness is considered by some to be the most common
interpretation of temporally unspecified relations in biology. We have
previously defined it in (\ref{eq:generically}), which can
now rephrased using the TCQ approach as follows:
\begin{equation}
\forall x:\; \mrelb{inst}(\TQC{A},x) \rightarrow \exists y :\;
\mrelb{inst}(\TQC{B}, y) \wedge \mrelb{rel}(x,y)
\label{eq:tqc:pg}
\end{equation}

Informally, this means that, whatever temporal qualification of an instance of
\mclass{A} we choose, it will alway be \mirel{rel}-related to some temporal
qualification of type \mclass{B}, but we neither care nor enforce which one.

This is the easiest and most elegant translation case from the FOL perspective.
Moving to OWL, the above axiom (\ref{eq:tqc:pg}) appears as: 

\begin{equation}
\TQC{A}\;\mathtt{subClassOf}\;\mrelb{rel}\;\mathtt{some}\;\TQC{B}
\label{eq:tqc:pg:owl}
\end{equation}

Again, we can use a kind of macro expansion to bridge a shorthand for this type
of relatedness (e.g. \mrelpg{rel}) back to the
underlying relationship. Thus we would use (\ref{eq:tqc:pg:owl}) to replace
every occurrence of axioms such as the following:
\begin{equation}
\mclass{A}\;\mathtt{subClassOf}\;\mrelpg{rel}\;\mathtt{some}\;\mclass{B}  
\label{eq:tqc:pg:shorthand}
\end{equation}

By replacing, we mean to imply that we advise against using \mrelpg{rel} as an
object property, which would not be harmful in itself, but at least
counter-intuitive. Instances of \mclass{A} satisfying
(\ref{eq:tqc:pg:shorthand} would require a pair of instances \pair{a}{b}, where
$a$ is said to be permanently generically related to $b$, which is (a)
meaningless since generic relatedness pertains to a type, not an instance and
(b) misleading since it is not enforced by the model.

Still, the availability of permanent generic relatedness is noteworthy because
it would not be possible in OWL 2 in absence of temporally qualified
continuants (not only instantiation of classes, but also of object property
tuples is rigid). This is afforded by the fact that we do not introduce an
explicit object property for generic permanent relatedness. 

Since permanent generic relatedness matches the presumed default interpretation
of \mrelb{rel} in most existing biomedical ontologies, upgrade paths for these
ontologies need to be considered. We believe that, from a user perspective, the
most convenient way would be to relegate all relations that require
temporalisation into a specific branch (say,
\mrelb{temporallySensitivelyRelated}) and employ something like the following 
preprocessing instruction:
\begin{lstlisting}
?x:CLASS[subClassOf Continuant],
?y:OBJECTPROPERTY?[subPropertyOf temporallySensitivelyRelated],
?z:CLASS[subClassOf Continuant] 
SELECT ?x subClassOf ?y some ?z WHERE 
  FAIL ?x subClassOf hasContinuant some Continuant,
  ?y MATCH("^(.(?<!temporarily_))*$")
  FAIL ?z subClassOf hasContinuant some Continuant,
BEGIN
  ADD hasContinuant some ?x subClassOf ?y some continuantOf some ?z,
  REMOVE ?x subClassOf ?y some ?z
END;
\end{lstlisting}
This allows users to do away with \mrelpg{rel} and have expressions about
continuants converted into expressions about temporal qualifications transformed
seamlessly. There are two downsides to this. Firstly, this approach moves
ontology engineering even more towards procedures that are familiar to
software engineers but not to scientists from the field of application.
Secondly, the above expression requires a reasoner to work, which might be
costly to do after every edit.

On the other hand, it is subject to debate whether adopting an edit-compile-test
approach in ontology engineering could in fact be useful for improving ontology quality.
\subsection{Permanent Specific Relatedness}
If we do introduce an explicit object property, could we hope to arrive at
implementing something like permanent specific relatedness
(\ref{eq:specifically})? Unfortunately, this assumption proves to be too naïve.

It would require an additional axiom to ensure that only TQCs of the same instance are involved
for the second relatum. Unfortunately, we cannot provide an
accurate translation of this kind of relatedness into OWL 2, though we can
achieve the following first order translation in TQC-talk:
 \begin{equation}
\begin{split}
\forall x:\; \mrelb{inst}&(\TQC{A},x) \rightarrow \\
 \exists y:&\;\mrelb{inst}(\TQC{B},y) \wedge \mrelb{rel}(x,y)\wedge\\
 & \forall x_1,a:\; ((\mrelb{inst}(\TQC{A},x_1) \wedge 
\mrelb{continuantOf}(a,x) \wedge \mrelb{continuantOf}(a,x_1))\wedge\\
&\;\;\exists y_1,b:\;(\mrelb{inst}(\TQC{B},y_1) \mrelb{continuantOf}(b,y)  
\wedge \mrelb{continuantOf}(b,y_1) \wedge\\&\;\;\;\mrelb{rel}(x_1,y_1)))
\end{split}
\end{equation}
This would require three variables ($x$,~$y$,~$y_1$) to be bound at the same
time, which is incompatible with any OWL translation.


\section{Conclusion}
We have shown that when expressing relationships between continuants, ontology
engineers cannot safely ignore the dynamic nature of reality without introducing
unnecessary ambiguity or even factual errors into ontologies. Since we
acknowledge the importance of temporary relatedness, we cannot simply accept an
informal reinterpretation that all OWL 2 relations imply permanent relatedness. 
We acknowledge that four-dimensionalist ontology is well suited for
dealing with this kind of problem. 

Still, we do not think that limitations of a certain formalism should force us
to adopt a four-dimensionalist world view. We agree that there are in fact
arguments in favour of four-dimensionalism that are worth discussing, but this is not one of them. 
Instead, we provide a formalisation that relies on making assertions about
temporally qualified continuants, representing continuants as they are viewed
within a restricted temporal context. This allows us to steer clear of the strong
commitment to exclusive four-dimensionalism, though of course the
four-dimensionalist is free to interpret TQCs as mere time slices.

We believe that the temporal qualification approach allows us to operate within
the confines of the ontological framework of BFO, where both occurrents and
continuants are useful and irreducible entities, thus providing a compelling
framework for existing users of this top-level ontology. One has, however, to
question the admissibility of temporal qualifications into a
realist top-level ontology, since they are obviously abstracted away from the
entire, changing, continuant. 

Regarding this problem, we believe the analogy with BFO 2 process profiles could 
make this approach at least sufferable from the
ontological perspective. Still, we admit that we have some reservations due to
an apparent multiplication of entities: At any given moment, the space occupied
by an operating table is not only occupied by the table itself, but also by it's
temporal qualification at that moment, and while one observes the table over a
period of time, those qualifications perish and come
into being incessantly, even if there is no apparent change in the table.  

If those reservations should prove to be overwhelming, we would remind ourselves
that the entire TQC scheme is strictly speaking a workaround for the reduced
expressivity of the language. If so, TQCs  need to be regarded as
unavoidable, technical artefacts, but they could be relegated to a specific
subtree of the ontology without disturbing the overall structure as much as a
outright transition to a four-dimensional solution.

As we have argued, special care has to be taken so that the procedures suggested
here are still manageable for the average ontology engineer. We
believe that the availability of macro expansion engines for ontology
development is key to this. While they do not add any additional expressivity, they
provide the ontology developer with succinct ways to express often used
concepts. In this case, there is virtually no onus on downstream users of the
ontology, since it is BFO policy to keep a fixed set of relations that is deemed
sufficient to express all ontological facts (where usually, the so called
"material" relations are written as class expressions because the correspond to
proper entities). 

There are thus two avenues for furthering this research: On the one hand, the
usability characteristics deserve detailed scrutiny, on the other hand, the
TQC approach have obvious connections to the (incomplete) theory of lives and
histories in BFO.
In the present paper, we have presented the crucial
relation \mrelb{continuantOf} as primitive and only given very little informal
explanation of its meaning. It would be desirable to give a more specific account of it in the
future, by using histories and projection relations between them and the
entities they are histories of.
\section*{Acknowledgements}
The research by NG, LJ and StS for this paper has been supported by the German
Science Foundation (DFG) , grant
JA1904/2-1, SCHU 2515/1-1 as part of the research project "Good Ontology Design".
%\bibliographystyle{natbib}
\printbibliography
\end{document}
